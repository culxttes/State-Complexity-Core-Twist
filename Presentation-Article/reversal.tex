% !chktex-file 8
% !chktex-file 41

\section{Renverser}

\setframetitle{Renverser}

\subsection{Préambule}

\begin{frame}{\myframetitle}
  \begin{block}{La complexité exacte du renverser}
    Nous allons montrer que l'opération de << renverser >> admet la complexité
    exacte de \(n + 1\) en considérant que l'automate résultant ne doit avoir
    qu'un seul état initial.
  \end{block}
\end{frame}

\subsection{Preuve}

\begin{frame}{\myframetitle}
  \begin{block}{Représentation de l'automate \(A\)}
    \centering
    \begin{tikzpicture}[
        scale=0.65,
        transform shape
      ]
      \node[state, accepting, initial] (0) {\(0\)};
      \node[state, accepting, right of=0] (1) {\(1\)};
      \node[state, accepting, draw=none, right of=1] (2) {\(\cdots\)};
      \node[state, accepting, right of=2] (3) {\(n - 2\)};
      \node[state, accepting, right of=3] (4) {\(n - 1\)};

      \draw
      (0) edge[above] node{\(a\)} (1)
      (1) edge[above] node{\(a\)} (2)
      (2) edge[above] node{\(a\)} (3)
      (3) edge[above] node{\(a\)} (4)
      (4) edge[above, bend right] node{\(b\)} (0);
    \end{tikzpicture}
  \end{block}

  \pause[]

  \begin{block}{\(M \in NFA(\Sigma)\)}
    Soit l'automate minimal \(M\) qui reconnait le miroir du langage de
    l'automate \(A\).
    % LTeX: enabled=false
    \begin{gather*}
      M = (Q, \{q_0\}, F, \delta)
    \end{gather*}
    % LTeX: enabled=true
  \end{block}
\end{frame}

\begin{frame}{\myframetitle}
  \setcounter{enumi}{1}
  \begin{block}{
      \raisebox{0.125\fontcharht\font`X}{%
        \resizebox{!}{\fontcharht\font`X}{\usebeamertemplate{enumerate item}}%
      }
      Preuve
    }
    \(M\) doit nécessairement reconnaitre le mot vide, car l'automate \(A\)
    reconnait lui-même le mot vide.

    \begin{center}
      \begin{tikzpicture}[
          scale=0.7,
          transform shape
        ]
        \node[state, accepting, initial] (0) {\(q_0\)};
        \node[state, draw=none, right of=0] (1) {};

        \draw (0) edge[dashed] node{} (1);
      \end{tikzpicture}
    \end{center}
  \end{block}
\end{frame}

\begin{frame}{\myframetitle}
  \setcounter{enumi}{2}
  \begin{block}{
      \raisebox{0.125\fontcharht\font`X}{%
        \resizebox{!}{\fontcharht\font`X}{\usebeamertemplate{enumerate item}}%
      }
      Preuve
    }
    L'automate \(A\) reconnait \(a^{n - 1}b\). Ainsi, il existe des états
    \(q_1, q_2, \ldots, q_n\).

    \begin{center}
      \begin{tikzpicture}[
          scale=0.7,
          transform shape
        ]
        \node[state, accepting, initial] (0) {\(q_0\)};
        \node[state, right of=0] (1) {\(q_1\)};
        \node[state, draw=none, right of=1] (2) {\(\cdots\)};
        \node[state, right of=2] (3) {\(q_{n - 1}\)};
        \node[state, accepting, right of=3] (4) {\(q_n\)};

        \draw
        (0) edge[above] node{\(b\)} (1)
        (1) edge[above] node{\(a\)} (2)
        (2) edge[above] node{\(a\)} (3)
        (3) edge[above] node{\(a\)} (4);

        \node[state, accepting, initial, below of=0] (0') {\(q_0\)};
        \node[state, right of=0'] (1') {\(q_1\)};
        \node[state, draw=none, right of=1'] (2') {\(\cdots\)};
        \node[state, right of=2'] (3') {\(q_{n - 1}\)};

        \draw
        (0') edge[above] node{\(b\)} (1')
        (1') edge[above] node{\(a\)} (2')
        (2') edge[above] node{\(a\)} (3')
        (3') edge[above, bend left] node{\(a\)} (0');
      \end{tikzpicture}
    \end{center}
  \end{block}
\end{frame}

\begin{frame}{\myframetitle}
  \setcounter{enumi}{3}
  \begin{block}{
      \raisebox{0.125\fontcharht\font`X}{%
        \resizebox{!}{\fontcharht\font`X}{\usebeamertemplate{enumerate item}}%
      }
      Preuve
    }
    \(M\) doit obligatoirement reconnaitre le mot \(a\), mais pas le mot \(b
    a^n\).

    \begin{center}
      \begin{tikzpicture}[
          scale=0.7,
          transform shape
        ]
        \node[state, accepting, initial] (0) {\(q_0\)};
        \node[state, right of=0] (1) {\(q_1\)};
        \node[state, draw=none, right of=1] (2) {\(\cdots\)};
        \node[state, right of=2] (3) {\(q_{n - 1}\)};
        \node[state, accepting, right of=3] (4) {\(q_n\)};

        \draw
        (0) edge[above] node{\(b\)} (1)
        (1) edge[above] node{\(a\)} (2)
        (2) edge[above] node{\(a\)} (3)
        (3) edge[above] node{\(a\)} (4)
        (0) edge[above, bend left] node{\(a\)} (4);
      \end{tikzpicture}
    \end{center}
  \end{block}
\end{frame}

\subsection{Conclusion}

\begin{frame}{\myframetitle}
  \begin{block}{Conclusion}
    Or, l'opération de renverser d'un langage rationnel à une complexité non
    déterministe de \(n + 1\).\footnotemark{}
  \end{block}
  \footnotetext{\hyperlink{Holzer2003}{\cite{Holzer2003}}}
\end{frame}