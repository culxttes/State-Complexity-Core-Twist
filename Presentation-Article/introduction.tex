% !chktex-file 8

\subsection{Présentation du papier}

\setframetitle{Présentation du papier}

\begin{frame}{\myframetitle}
  \begin{block}{Présentation du papier}
    % LTex: language=en
    \textit{State complexity of some operations on binary regular
    languages}\footnotemark{}
    % LTex: language=fr

    \vphantom{}

    Par \textbf{Galina Jirásková}, publié dans
    % LTex: language=en
    \textit{\underline{Theoretical Computer Science, 330.2}}
    % LTex: language=fr%
    en 2005
  \end{block}
  \footnotetext{\hyperlink{JIRASKOVA2005287}{\cite{JIRASKOVA2005287}}}
\end{frame}

\section{Introduction}

\subsection{Prémisse}

\subsubsection{Alphabet et Mot}

\setframetitle{Alphabet et Mot}

\begin{frame}{\myframetitle}
  \begin{definition}[Alphabet]
    Un << alphabet >> \(\Sigma\) est un ensemble fini non-vide de symboles.
  \end{definition}

  \pause[]

  \begin{definition}[Mot]
    Un << mot >> \(w\) est une suite finie de symboles sur un alphabet
    \(\Sigma\). La suite vide de symbole sera notée par \(\varepsilon\).
  \end{definition}

  \pause[]

  \begin{example}
    \centering
    \vspace{-1.5\topsep}
    \begin{gather*}
      \Sigma = \{a, b\} \\
      w = abba
    \end{gather*}
  \end{example}
\end{frame}

\begin{frame}{\myframetitle}
  \begin{definition}[Concaténation d'un mot]
    L'opération \(\cdot\) désigne la << concaténation >> de deux mots.
  \end{definition}

  \pause[]

  \begin{example}
    \centering
    \vspace{-1.5\topsep}
    \begin{gather*}
      u = aa \text{ et } v = bb \\
      w = u \cdot v = aabb
    \end{gather*}
  \end{example}
\end{frame}

\begin{frame}{\myframetitle}
  \begin{definition}[Le miroir d'un mot]
    On notera \(\overleftarrow w\), le << miroir >> du mot \(w\).
  \end{definition}

  \pause[]

  \begin{example}
    \vspace{-1.5\topsep}
    \begin{align*}
      u &= abcd\\
      \overleftarrow u &= dcba
    \end{align*}
  \end{example}
\end{frame}

\subsubsection{Langage}

\setframetitle{Langage}

\begin{frame}{\myframetitle}
  \begin{definition}[Langage]
    Un << langage >> \(L\) est un ensemble de mots sur un alphabet \(\Sigma\).
  \end{definition}

  \pause[]

  \begin{example}
    \vspace{-1.5\topsep}
    \begin{gather*}
      L = \{\varepsilon, aa, bb, abab\}
    \end{gather*}
  \end{example}
\end{frame}

\begin{frame}{\myframetitle}
  \begin{definition}[La concaténation]
    La << concaténation >> est définie grâce à la concaténation des mots~:

    \begin{alignat*}{2}
      &L_1 &&= \{a, b\} \\
      &L_2 &&= \{c, d\} \\
      &L_1 \cdot L_2 &&= \{ac, ad, bc, bd\} \\
    \end{alignat*}
  \end{definition}
\end{frame}

\begin{frame}{\myframetitle}
  \begin{definition}[La copie n-ième]
    On définit la << copie n-ième >> d'un langage \(L\) noté \(L^n\)~:

    \begin{alignat*}{2}
      &L_1 &&= \{a, b\} \\
      &L_1^2 &&= \{aa, ab, bb, ba\}
    \end{alignat*}
  \end{definition}
\end{frame}

\begin{frame}{\myframetitle}
  \begin{definition}[L'étoile (de \textit{Kleene}) d'un langage]
    On peut définir << l'étoile >> d'un langage notée \(L^*\)~:
    % LTeX: enabled=false
    \begin{align*}
      L^* = \bigcup_{i \geq 0} L^i
    \end{align*}
    % LTeX: enabled=true
  \end{definition}

  \pause[]

  \begin{example}
    \vspace{-1.5\topsep}
    \begin{align*}
      &L = \{a, b\} \\
      &L^* = \{\varepsilon, a, b, aa, ab, \ldots\}
    \end{align*}
  \end{example}
\end{frame}

\begin{frame}{\myframetitle}
  \begin{definition}[Le langage de l'ensemble des mots]
    L'ensemble des mots possible sur l'alphabet \(\Sigma\) sera noté
    \(\Sigma^*\).
  \end{definition}

  \pause[]

  \begin{definition}[Le langage miroir]
    On pourra noter \(\overleftarrow L\), << le langage miroir >> de \(L\), qui
    sera le langage des miroirs des mots de \(L\)~:
    \begin{align*}
      \overleftarrow L = \{\overleftarrow w ~|~ w \in L\}
    \end{align*}
  \end{definition}

  \pause[]

  \begin{block}{Autres opérations}
    Les langages étant des ensembles, leurs opérations peuvent être étendues
    aux langages.
  \end{block}
\end{frame}

\begin{frame}{\myframetitle}
  \begin{definition}[Les langages rationnels]
    L'ensemble des << langages rationnels >> sur \(\Sigma\), noté
    \(\RAT(\Sigma^*)\), est le plus petit ensemble de langages sur \(\Sigma\)
    qui vérifie~:

    \vphantom{}

    \begin{itemize}
      \item \(\RAT(\Sigma^*)\) contient \(\varnothing\) et \(\{a\}\) avec \(a
        \in \Sigma\).
      \item \(\RAT(\Sigma^*)\) est fermé pour l'union, la concaténation et
        l'étoile.
    \end{itemize}
  \end{definition}
\end{frame}

\subsubsection{Automate}

\setframetitle{Automate}

\begin{frame}{\myframetitle}
  \begin{definition}[Automate]
    \(M = (Q, I, F, \delta)\) avec \(Q = \{q_1, q_2, q_3\}\), \(I = \{q_1\}\) et
    \(F = \{q_3\}\).
    \begin{center}
      \begin{tikzpicture}
        \node[state, initial] (q1) {\(q_1\)};
        \node[state, right of=q1] (q2) {\(q_2\)};
        \node[state, accepting, right of=q2] (q3) {\(q_3\)};

        \draw
        (q1) edge[loop above] node{\(a\)} (q1)
        (q1) edge[above] node{\(b\)} (q2)
        (q1) edge[bend right, below] node{\(a\)} (q3)
        (q2) edge[above] node{\(a,b\)} (q3)
        (q3) edge[bend right=50, above] node{\(b\)} (q2);
      \end{tikzpicture}
    \end{center}
  \end{definition}
\end{frame}

\begin{frame}{\myframetitle}
  \begin{definition}[L'acceptation d'un mot]
    Le mot \(aba\) est << accepté >>~:
    \begin{center}
      \begin{tikzpicture}
        \node[state, initial] (q1) {\(q_1\)};
        \node[state, right of=q1] (q2) {\(q_2\)};
        \node[state, accepting, right of=q2] (q3) {\(q_3\)};

        \only<1> {\draw (q1) edge[loop above] node{\(a\)} (q1);}
        \only<1-2> {\draw (q1) edge[above] node{\(b\)} (q2);}
        \only<1-3> {\draw (q2) edge[above] node{\(a,b\)} (q3);}

        \only<2-> {\draw (q1) edge[loop above, red] node{\(a\)} (q1);}
        \only<3-> {\draw (q1) edge[above, red] node{\(b\)} (q2);}
        \only<4-> {\draw (q2) edge[above, red] node{\(a,b\)} (q3);}

        \draw
        (q3) edge[bend right=50, above] node{\(b\)} (q2)
        (q1) edge[bend right, below] node{\(a\)} (q3);
      \end{tikzpicture}
    \end{center}
  \end{definition}
\end{frame}

\begin{frame}{\myframetitle}
  \begin{definition}[Automate déterministe]
    On parlera d'automate << déterministe >> quand~:
    \begin{center}
      \only<1> {
        \begin{tikzpicture}
          \node[state, initial] (q1) {\(q_1\)};
          \node[state, right of=q1] (q2) {\(q_2\)};
          \node[state, accepting, right of=q2] (q3) {\(q_3\)};

          \draw
          (q1) edge[loop above] node{\(a\)} (q1)
          (q1) edge[above] node{\(b\)} (q2)
          (q2) edge[above] node{\(a,b\)} (q3)
          (q3) edge[bend right=50, above] node{\(b\)} (q2)
          (q3) edge[bend left, below] node{\(a\)} (q1);
        \end{tikzpicture}
      }

      \only<2> {
        \begin{tikzpicture}[every initial by arrow/.style={red}]
          \node[state, initial] (q1) {\(q_1\)};
          \node[state, right of=q1] (q2) {\(q_2\)};
          \node[state, accepting, right of=q2] (q3) {\(q_3\)};

          \draw
          (q1) edge[loop above] node{\(a\)} (q1)
          (q1) edge[above] node{\(b\)} (q2)
          (q2) edge[above] node{\(a,b\)} (q3)
          (q3) edge[bend right=50, above] node{\(b\)} (q2)
          (q3) edge[bend left, below] node{\(a\)} (q1);
        \end{tikzpicture}
      }

      \only<3> {
        \begin{tikzpicture}[every initial by arrow/.style={gray}]
          \node[state, initial] (q1) {\(q_1\)};
          \node[state, right of=q1, gray] (q2) {\(q_2\)};
          \node[state, accepting, right of=q2, gray] (q3) {\(q_3\)};

          \draw
          (q1) edge[loop above] node{\(a\)} (q1)
          (q1) edge[above] node{\(b\)} (q2)
          (q2) edge[above, gray] node{\(a,b\)} (q3)
          (q3) edge[bend right=50, above, gray] node{\(b\)} (q2)
          (q3) edge[bend left, below, gray] node{\(a\)} (q1);
        \end{tikzpicture}
      }

      \only<4> {
        \begin{tikzpicture}[every initial by arrow/.style={gray}]
          \node[state, initial, gray] (q1) {\(q_1\)};
          \node[state, right of=q1] (q2) {\(q_2\)};
          \node[state, accepting, right of=q2, gray] (q3) {\(q_3\)};

          \draw
          (q1) edge[loop above, gray] node{\(a\)} (q1)
          (q1) edge[above, gray] node{\(b\)} (q2)
          (q2) edge[above] node{\(a,b\)} (q3)
          (q3) edge[bend right=50, above, gray] node{\(b\)} (q2)
          (q3) edge[bend left, below, gray] node{\(a\)} (q1);
        \end{tikzpicture}
      }

      \only<5> {
        \begin{tikzpicture}[every initial by arrow/.style={gray}]
          \node[state, initial, gray] (q1) {\(q_1\)};
          \node[state, right of=q1, gray] (q2) {\(q_2\)};
          \node[state, accepting, right of=q2] (q3) {\(q_3\)};

          \draw
          (q1) edge[loop above, gray] node{\(a\)} (q1)
          (q1) edge[above, gray] node{\(b\)} (q2)
          (q2) edge[above, gray] node{\(a,b\)} (q3)
          (q3) edge[bend right=50, above] node{\(b\)} (q2)
          (q3) edge[bend left, below] node{\(a\)} (q1);
        \end{tikzpicture}
      }
    \end{center}
  \end{definition}
\end{frame}

\begin{frame}{\myframetitle}
  \begin{definition}[Automate minimal]
    L'automate \(M_1\) est << minimal >> du langage \(L = \{ab\}\)~:

    \begin{minipage}{0.47\textwidth}
      \centering
      \begin{tikzpicture}[node distance=2.5cm]
        \node[state, initial] (0) {\(0\)};
        \node[state, right of=0] (1) {\(1\)};
        \node[state, accepting, below of=1] (2) {\(2\)};

        \node[above =.75cm of $(0)!0.5!(1)$, align=center] {\(M_1\)};

        \draw
        (0) edge[above] node{\(a\)} (1)
        (1) edge[right] node{\(b\)} (2);
      \end{tikzpicture}
    \end{minipage}
    \hfill
    \begin{minipage}{0.47\textwidth}
      \centering
      \begin{tikzpicture}[node distance=2.5cm]
        \node[state, initial] (0) {\(0\)};
        \node[state, right of=0] (1) {\(1\)};
        \node[state, accepting, below of=1] (2) {\(2\)};
        \node[state, left of=2] (3) {\(3\)};

        \node[above =.75cm of $(0)!0.5!(1)$, align=center] {\(M_2\)};

        \draw
        (0) edge[above] node{\(a\)} (1)
        (1) edge[right] node{\(b\)} (2)
        (0) edge[left] node{\(a\)} (3);
      \end{tikzpicture}
    \end{minipage}
  \end{definition}
\end{frame}

\subsubsection{Lien entre les langages et les automates}

\setframetitle{Lien entre les langages et les automates}

\begin{frame}{\myframetitle}
  \begin{definition}[L'ensemble des langages reconnaisable]
    On définira l'ensemble des langages de \(\Sigma^*\) << reconnaissable >> par
    au moins un automate par \(\REC(\Sigma^*)\).
  \end{definition}

  \pause[]

  \begin{block}{Théorème de \textit{Kleene}}
    L'informaticien \textit{Stephen C. Kleene} a montré en 1956 que~:
    % LTeX: enabled=false
    \begin{align*}
      \REC(\Sigma^*) = \RAT(\Sigma^*)
    \end{align*}
    % LTeX: enabled=true
  \end{block}
\end{frame}

\subsubsection{Complexité en état}

\setframetitle{Complexité en état}

\begin{frame}{\myframetitle}
  \begin{definition}
    La << complexité en état >> est une mesure d'un \textbf{langage}, elle est
    définie comme le nombre d'états de l'automate minimal du langage~:

    \vphantom{}

    \begin{itemize}
      \item Complexité en état déterministe (qu'on notera \(C_{\det}\)).
      \item Complexité en état non déterministe (qu'on notera \(C_{\ndet}\)).
    \end{itemize}
  \end{definition}
\end{frame}

\begin{frame}{\myframetitle}
  \begin{example}[%
      \only<1>{Définition d'une famille}%
      \only<2>{\(C_{\ndet}(A_2) = 2 + 2\)}%
      \only<3>{\(C_{\det}(A_2) = 2^{2 + 1}\)}%
    ]
    \centering
    \vspace{-1.5\topsep}
    \begin{gather*}
      A_n = \Sigma^* \cdot a \cdot \Sigma^n
    \end{gather*}
    \vspace{-2.25\topsep}

    \onslide<2-> {
      \begin{tikzpicture}[node distance=2.5cm, scale=0.7,transform shape]
        \node[state, initial] (0') {\(0\)};
        \node[state, right of=0'] (1') {\(1\)};
        \node[state, below of=1'] (2') {\(2\)};
        \node[state, accepting, left of=2'] (3') {\(3\)};

        \draw
        (0') edge[above] node{\(a\)} (1')
        (1') edge[right] node{\(a,b\)} (2')
        (2') edge[above] node{\(a,b\)} (3')
        (0') edge[loop above] node{\(a,b\)} (0');

        \onslide<3> {
          \node[state, initial, right =5cm of 1'] (0) {\(0\)};
          \node[state, right of=0] (1) {\(1\)};
          \node[state, right of=1] (2) {\(2\)};
          \node[state, right of=2] (3) {\(3\)};
          \node[state, accepting, below of=3] (4) {\(4\)};
          \node[state, accepting, left of=4] (5) {\(5\)};
          \node[state, accepting, left of=5] (6) {\(6\)};
          \node[state, accepting, left of=6] (7) {\(7\)};

          \draw
          (0) edge[loop above] node{\(b\)} (0)
          (3) edge[loop above] node{\(a\)} (3)

          (0) edge[above] node{\(a\)} (1)
          (1) edge[above] node{\(a\)} (2)
          (2) edge[above] node{\(a\)} (3)
          (3) edge[right] node{\(b\)} (4)
          (4) edge[above] node{\(a\)} (5)
          (5) edge[above] node{\(b\)} (6)
          (6) edge[above] node{\(b\)} (7)

          (1) edge[left] node{\(b\)} (6)
          (2) edge[left] node{\(b\)} (4)
          (4) edge[below, bend left] node{\(b\)} (7)
          (5) edge[left] node{\(a\)} (2)
          (6) edge[above, bend right] node{\(a\)} (5)
          (7) edge[left] node{\(a\)} (1)
          (7) edge[left] node{\(b\)} (0);
        }
      \end{tikzpicture}
    }
  \end{example}
\end{frame}

\begin{frame}{\myframetitle}
  \begin{definition}[Complexité en état opérationnielle]
    La fonction \(f\) à une complexité \(g(n, m)\) si~:
    \begin{center}
      \begin{tikzpicture}[transform shape, scale=0.8]
        \node[draw, ellipse, minimum width=1.5cm] (A) at (-2,2) {\(A\)};
        \node[draw, ellipse, minimum width=1.5cm] (B) at (2,2) {\(B\)};
        \node[draw, rectangle, minimum width=2cm, minimum height=1cm]
        (F) at (0,0) {\(f\)};
        \node[draw, ellipse, minimum width=1.5cm] (C) at (0,-2) {\(C\)};

        \node[left =0.25cm of A] {\(n\)};
        \node[right =0.25cm of B] {\(m\)};
        \node[below =0.25cm of C] {\(k \leq g(n, m)\)};

        \coordinate (center) at ($(F.north west)!0.5!(F.north east)$);

        \draw (A) -- (-0.5,2) -- ([shift={(-0.5cm,0cm)}] center);
        \draw (B) -- (0.5,2) -- ([shift={(0.5cm,0cm)}] center);
        \draw (F) -- (C);
      \end{tikzpicture}
    \end{center}
  \end{definition}
\end{frame}