% !chktex-file 8
% !chktex-file 41

\appendix

\section{Annexe}

\setframetitle{Pire cas de la construction par sous-ensemble}

\subsection{Pire cas de la construction par sous-ensemble}

\subsubsection{Préambule}

\begin{frame}{\myframetitle}\label{pcse}
  \begin{block}{La méthode appliquée}
    Prenons un automate non déterministe minimal et prouvons que sa version
    déterministe à \(2^n\) états.
  \end{block}
\end{frame}

\subsubsection{Preuve}

\begin{frame}{\myframetitle}
  \begin{block}{\(A \in \NFA(\Sigma, \mathbb{N})\)}
    \(A = (Q_A, \{0\}, \{0\}, \delta_A)\) avec~:
    % LTeX: enabled=false
    \begin{gather*}
      \forall i \in \llbracket 0, n - 1 \rrbracket, \\
      \delta_A(i, \sigma) =
      \begin{cases}
        \{i + 1\} &\text{si } i < n - 1 \text{ et } \sigma = a, \\
        \{0\} &\text{si } i = n - 1 \text{ et } \sigma = a, \\
        \{0, i\} &\text{si } i > 0 \text{ et } \sigma = b \\
      \end{cases}
    \end{gather*}
    % LTeX: enabled=true
  \end{block}
\end{frame}

\begin{frame}{\myframetitle}
  \begin{block}{Représentation de l'automate \(A\)}
    % LTeX: enabled=false
    \centering
    \begin{tikzpicture}[
        scale=0.65,
        transform shape,
        every state/.style={
          circle,
          minimum size=1.25cm,
          inner sep=0
        }
      ]

      \pgfmathsetmacro{\w}{8}
      \pgfmathsetmacro{\h}{3}
      \pgfmathsetmacro{\s}{270}
      \pgfmathsetmacro{\a}{360/7}

      \node[state, accepting, initial below] (0)
      at ({\w * cos(\a * 0 + \s)}, {\h * sin(\a * 0 + \s)}) {\(0\)};

      \node[state] (1)
      at ({\w * cos(\a * 1 + \s)}, {\h * sin(\a * 1 + \s)}) {\(1\)};

      \node[state] (2)
      at ({\w * cos(\a * 2 + \s)}, {\h * sin(\a * 2 + \s)}) {\(2\)};

      \node[state, draw=none] (3)
      at ({\w * cos(\a * 3 + \s)}, {\h * sin(\a * 3 + \s)}) {\(\cdots\)};

      \node[state, draw=none] (4)
      at ({\w * cos(\a * 4 + \s)}, {\h * sin(\a * 4 + \s)}) {\(\cdots\)};

      \node[state] (5)
      at ({\w * cos(\a * 5 + \s)}, {\h * sin(\a * 5 + \s)}) {\(n - 2\)};

      \node[state] (6)
      at ({\w * cos(\a * 6 + \s)}, {\h * sin(\a * 6 + \s)}) {\(n - 1\)};

      \draw
      (1) edge[loop below] node{\(b\)} (1)
      (2) edge[loop right] node{\(b\)} (2)
      (3) edge[loop above] node{\(b\)} (3)
      (4) edge[loop above] node{\(b\)} (4)
      (5) edge[loop left] node{\(b\)} (5)
      (6) edge[loop below] node{\(b\)} (6)

      (1) edge[bend right={\a * 0.5}, below] node{\(b\)} (0)
      (2) edge[bend right={\a * 0.5}, below] node{\(b\)} (0)
      (3) edge[bend right={\a * 0.5}, below] node{\(b\)} (0)
      (4) edge[bend left={\a * 0.5}, below] node{\(b\)} (0)
      (5) edge[bend left={\a * 0.5}, below] node{\(b\)} (0)
      (6) edge[bend left={\a * 0.5}, below] node{\(b\)} (0)

      (0) edge[bend right={\a * 0.2}, above] node{\(a\)} (1)
      (1) edge[bend right={\a * 0.5}, left] node{\(a\)} (2)
      (2) edge[bend right={\a * 0.2}, below] node{\(a\)} (3)
      (3) edge[bend right={\a * 0.2}, below] node{\(a\)} (4)
      (4) edge[bend right={\a * 0.2}, below] node{\(a\)} (5)
      (5) edge[bend right={\a * 0.5}, right] node{\(a\)} (6)
      (6) edge[bend right={\a * 0.2}, above] node{\(a\)} (0);
    \end{tikzpicture}
    % LTeX: enabled=true
  \end{block}
\end{frame}

\begin{frame}{\myframetitle}
  \begin{block}{Logique appliquée}
    Pour démontrer que l'automate déterministe minimal de \(A\), noté \(M\),
    possède \(2^n\) états, nous allons prouver que l'ensemble \(2^{Q_A}\)
    correspond aux états accessibles et qu'aucun de ces états n'est équivalent à
    un autre.
  \end{block}
\end{frame}

\begin{frame}{\myframetitle}
  \begin{block}{Convention de notation}
    Nous allons définir~:
    % LTeX: enabled=false
    \begin{align*}
      &B_p = (b_{n - 1}, b_{n - 2}, \ldots, b_{1}, b_{0}) \\
      &b_i =
      \begin{cases}
        0 & \text{si } i \in p, \\
        1 & \text{si } i \notin p.
      \end{cases}
      \text{ et } p \in 2^{Q_A}
    \end{align*}
    % LTeX: enabled=true
    On remarquera que les suites \(B_p\) peuvent être vu comme des nombres
    binaires.
  \end{block}
\end{frame}

\begin{frame}{\myframetitle}
  \begin{block}{Remarque}
    Avec ça on peut remarquer que pour tout nombre \(B_p\)~:
    % LTeX: enabled=false
    \begin{gather*}
      \delta_M(B_p, \sigma) =
      \begin{cases}
        RotateLeft(B_p) &\text{si } p \neq \varnothing \text{ et }
        \sigma = a, \\
        LSB(B_p, 1) &\text{si } p \neq \varnothing \land p \neq \{1\}
        \text{ et } \sigma = b, \\
        B_{\varnothing} &\text{si } p = \{1\} \text{ et } \sigma = b, \\
        B_{\varnothing} &\text{si } p = \varnothing.
      \end{cases}
    \end{gather*}
    % LTeX: enabled=true
    Où \(RotateLeft(w)\) est une rotation circulaire vers la gauche des bits de
    \(w\) et \(LSB(w, v)\) met le bit de point faible de \(w\) à \(v\).
  \end{block}
\end{frame}

\begin{frame}{\myframetitle}
  \begin{block}{Remarque}
    L'état \(0 \ldots 1 \) est accessible, car il correspond à l'état initial.
    De même, l'état \(0 \ldots 0\) est également accessible, puisqu'en lisant un
    \(b\) depuis \(0 \ldots 1\), on y parvient.
  \end{block}

  \pause[]

  \begin{block}{Preuve par induction}
    Nous souhaitons démontrer que tous les nombres de \(0\) à \(2^n\) peuvent
    être générés avec uniquement les nombres \(0\) et \(1\), ainsi que les
    opérations \(RotateLeft\) et \(LSB\). Pour cela, nous allons effectuer une
    preuve par récurrence fondée sur la valeur du nombre \(k\).
  \end{block}
\end{frame}

\begin{frame}{\myframetitle}
  \begin{block}{Base~: \(k = 0\) et \(k = 1\)}
    La base de la récurrence est vérifiée. En effet, il est trivial de générer
    les nombres \(0\) et \(1\) à partir des valeurs \(0\) et \(1\)
    respectivement.
  \end{block}
\end{frame}

\begin{frame}{\myframetitle}
  \begin{block}{Induction~: \(k \geq 2\)}
    \begin{itemize}
      \item Si \(k\) est pair, alors~:
        \[
          k = 2 k' \Longrightarrow k = RotateLeft(k')
        \]
      \item Si \(k\) est impair, ainsi~:
        \[
          k = k' + 1 \Longrightarrow k = LSB(k', 1)
        \]
    \end{itemize}
    Par hypothèse de récurrence, \(k'\) peut être généré. Par conséquent, \(k\)
    peut également être généré.
  \end{block}
\end{frame}

\begin{frame}{\myframetitle}
  \begin{block}{Preuve par induction}

    Nous avons démontré, par récurrence, que tous les nombres de \(0\) à \(2^n\)
    peuvent être générés avec uniquement les deux opérations (\(RotateLeft\) et
    \(LSB\)) ainsi que les deux nombres (\(0\) et \(1\)).
  \end{block}

  \pause[]

  \begin{block}{\(\forall (p, q) \in 2^{Q_A}, p \nLeftrightarrow q\)}
    Il nous suffit donc maintenant de prouver qu'aucun état n'est équivalent à
    un autre.
  \end{block}
\end{frame}

\begin{frame}{\myframetitle}
  \begin{block}{Preuve}
    Soit \(S \in 2^{Q_A}\) et \(T \in 2^{Q}\) distinct, alors sans perte de
    généralité, il existe un \(p\) qui est dans \(S\) et pas dans \(T\). Ainsi,
    en partant de \(S\), le mot \(a^{n - p}\) est accepté, mais pas en partant
    de \(T\).
  \end{block}
\end{frame}

\subsubsection{Conclusion}

\begin{frame}{\myframetitle}
  \begin{block}{L'automate \(M\) est minimal}
    Par conséquent, l'automate déterministe \(M\) est minimal et à \(2^n\)
    états.
  \end{block}
\end{frame}