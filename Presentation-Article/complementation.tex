% !chktex-file 8
% !chktex-file 41

\section{Calcul du complémentaire}

\setframetitle{Calcul du complémentaire}

\subsection{Préambule}

\begin{frame}{\myframetitle}
  \begin{block}{La complexité du calcul du complémentaire}
    \textit{Galina Jirásková} a démontré~\cite{JIRASKOVA2005287} que le
    \blockquote{calcul du complémentaire} a une complexité exacte de
    \(O_{\ndet}(2^n)\).
  \end{block}
\end{frame}

\begin{frame}{\myframetitle}
  \begin{definition}[L'ensemble piégeant]
    Un ensemble de couple de mots \(\{(x_i, y_i) \mid i \in \llbracket 0, n - 1
    \rrbracket\}\) est dit un << ensemble piégeant >> pour un langage rationnel
    \(L\). Si pour \(\forall i \in \llbracket 0, n - 1 \rrbracket\)~:
    \begin{itemize}
      \item \(x_i \cdot y_j \in L\).
      \item \(i \neq j \Longrightarrow x_i \cdot y_j \notin L \lor
        x_j \cdot y_i \notin L\).
    \end{itemize}
  \end{definition}

  \pause[]

  \begin{lemma}[%
      de \textit{Jean-Camille Birget}~\cite{BIRGET1992185}\label{lem:birget}%
    ]
    Soit l'ensemble piégeant \(\{(x_i, y_i) \mid i \in \llbracket 0, n - 1
    \rrbracket\}\) du langage rationnel \(L\). Alors, l'automate non
    déterministe minimal qui reconnait \(L\) à minimum \(n\) états.
  \end{lemma}
\end{frame}

\begin{frame}{\myframetitle}
  \begin{block}{La méthode appliquée}
    Nous allons prendre un automate minimal non déterministe à \(n\) états et
    prouver que le langage complémentaire du langage reconnu par l'automate a un
    ensemble piégeant de cardinal \(2^n\).
  \end{block}
\end{frame}

\subsection{Preuve}

% \begin{frame}{\myframetitle}
%   \begin{block}{\(M \in \NFA(\Sigma, \mathbb{N})\)}
%     \(M = (Q, \{0\}, \{n - 1\}, \delta)\), \(\Sigma = \{0, 1\}\),
%     \(L = L(M)\) avec~:
%     \begin{gather*}
%       \forall i \in \llbracket 0, n - 1 \rrbracket, \\
%       \delta(i, \sigma) =
%       \begin{cases}
%         \{i + 1\} &\text{si } i < n - 1 \text{ et } \sigma = 1, \\
%         \varnothing &\text{si } i = n - 1 \text{ et } \sigma = 1, \\
%         \{0, i + 1\} &\text{si } i < n - 1 \text{ et } \sigma = 0, \\
%         Q \setminus \{0\} &\text{si } i = n - 1 \text{ et }
%         \sigma = 0. \\
%       \end{cases}
%     \end{gather*}
%   \end{block}
% \end{frame}

\begin{frame}{\myframetitle}
  \begin{block}{Représentation de l'automate \(M\)}
    \centering
    \begin{tikzpicture}[
        scale=0.7,
        transform shape
      ]
      \node[state, initial] (0) {\(0\)};
      \node[state, right of=0] (1) {\(1\)};
      \node[state, right of=1] (2) {\(2\)};
      \node[state, draw=none, right of=2] (3) {\(\cdots\)};
      \node[state, right of=3] (4) {\(n - 2\)};
      \node[state, accepting, right of=4] (5) {\(n - 1\)};

      \draw
      (0) edge[loop above] node{\(0\)} (0)
      (5) edge[loop above] node{\(0\)} (5)

      (0) edge[above] node{\(0,1\)} (1)
      (1) edge[above] node{\(0,1\)} (2)
      (2) edge[above] node{\(0,1\)} (3)
      (3) edge[above] node{\(0,1\)} (4)
      (4) edge[above] node{\(0,1\)} (5)

      (1) edge[above, bend left=40] node{\(0\)} (0)
      (2) edge[above, bend left=50] node{\(0\)} (0)
      (4) edge[above, bend left=60] node[xshift=-1cm]{\(0\)} (0)

      (5) edge[above, bend left=40] node{\(0\)} (4)
      (5) edge[above, bend left=50] node[xshift=-2cm, yshift=0.4cm]{\(0\)} (2)
      (5) edge[above, bend left=60] node[xshift=1cm]{\(0\)} (1);
    \end{tikzpicture}
  \end{block}
\end{frame}

\begin{frame}{\myframetitle}
  \setcounter{enumi}{1}
  \begin{block}{
      \raisebox{0.125\fontcharht\font`X}{%
        \resizebox{!}{\fontcharht\font`X}{\usebeamertemplate{enumerate item}}%
      }
      Preuve
    }
    On va fixer un ensemble~:
    \begin{gather*}
      S = \left(\bigcup\limits_{i = 0}^{n - 1} \Sigma^i\right) \cup \{1^n\}
    \end{gather*}
    Pour tout mot \(x\) dans \(S\), on définit un mot \(y_x\) tel que
    \(\{(x, y_x \mid x \in S)\}\) est un ensemble piégeant.
  \end{block}
\end{frame}

\begin{frame}{\myframetitle}
  \begin{block}{
      \raisebox{0.125\fontcharht\font`X}{%
        \resizebox{!}{\fontcharht\font`X}{\usebeamertemplate{enumerate item}}%
      }
      Preuve
    }
    \vspace{-1.5\topsep}
    % LTeX: enabled=false
    \begin{gather*}
      \text{Pour } x \in S \text{ alors~:}\\
      y_x =
      \begin{cases}
        1^n &\text{si } \delta(0, x) = Q, \\
        \varepsilon &\text{si } \delta(0, x) = Q \setminus \{n - 1\}.
      \end{cases}
    \end{gather*}
    % LTeX: enabled=true
  \end{block}
\end{frame}

\begin{frame}{\myframetitle}
  \begin{block}{
      \raisebox{0.125\fontcharht\font`X}{%
        \resizebox{!}{\fontcharht\font`X}{\usebeamertemplate{enumerate item}}%
      }
      Preuve
    }
    Sinon \(y_x = y_1 \cdot y_2 \cdots y_{n - 1 - k}\) avec \(k\), le plus petit
    nombre dans \(Q\) qui n'était pas dans \(\delta(0, x)\) et pour \(j \in
    \llbracket 1, n - 1 - k \rrbracket\)~:
    \begin{gather*}
      y_j =
      \begin{cases}
        1 &\text{si } n - j \in \delta(0, x), \\
        0 &\text{si } n - j \notin \delta(0, x).
      \end{cases}
    \end{gather*}
  \end{block}
\end{frame}

\begin{frame}{\myframetitle}
  \begin{lemma}[%
      de \textit{Galina Jirásková}~\cite{JIRASKOVA2005287}\label{lem:g1}%
    ]
    \vspace{-1.5\topsep}
    \begin{gather*}
      \forall x \in S, \forall p \in Q \\
      p \in \delta(0, x) \Longrightarrow n - 1 \notin \delta(p, y_x) \\
      \lor \\
      p \notin \delta(0, x) \Longrightarrow n - 1 \in \delta(p, y_x)
    \end{gather*}
  \end{lemma}

  \pause[]

  \begin{lemma}[%
      de \textit{Galina Jirásková}~\cite{JIRASKOVA2005287}\label{lem:g2}%
    ]
    \vspace{-1.5\topsep}
    \begin{gather*}
      \forall (x, z) \in S^2, x \neq z \Longrightarrow \delta(0, x) \neq
      \delta(0, z)
    \end{gather*}
  \end{lemma}
\end{frame}

\begin{frame}{\myframetitle}
  \setcounter{enumi}{2}
  \begin{block}{
      \raisebox{0.125\fontcharht\font`X}{%
        \resizebox{!}{\fontcharht\font`X}{\usebeamertemplate{enumerate item}}%
      }
      Preuve
    }
    Pour \(x\) dans \(S\), l'état \(n - 1\) n'est pas accessible depuis aucun
    état de \(\delta(0, x)\) en lisant le mot \(y_x\) (Lemme~\ref{lem:g1}). Or
    l'état \(n - 1\) est le seul état final de \(M\), donc le mot \(x \cdot
    y_x\) n'est pas dans \(L(M)\).
  \end{block}

  \pause[]

  \setcounter{enumi}{3}
  \begin{block}{
      \raisebox{0.125\fontcharht\font`X}{%
        \resizebox{!}{\fontcharht\font`X}{\usebeamertemplate{enumerate item}}%
      }
      Preuve
    }
    Soit \(x\) et \(z\), deux mots distincts dans \(S\), sans perte de
    généralité, il existe un \(q\) tel que \(q \in \delta(0, x)\) et \(q \notin
    \delta(0, z)\) (Lemme~\ref{lem:g2}). Par conséquent, l'état \(n - 1\) peut
    être accédé depuis \(q\) en lisant \(y_z\) (Lemme~\ref{lem:g1}).
  \end{block}
\end{frame}

\subsection{Conclusion}

\begin{frame}{\myframetitle}
  \begin{block}{Conclusion}
    On vient donc de prouver que l'ensemble \(\{(x, y_x) \mid x \in S\}\) est
    un ensemble piégeant de taille \(2^n\). Ainsi, l'automate minimal doit au
    moins avoir \(2^n\) états (Lemme~\ref{lem:birget}).

    \onslide<2> {
      \vphantom{}

      Or \textit{Michael O Rabin} et \text{Dana S. Scott} ont
      prouvé~\cite{RS1959} que le calcul du complémentaire pour un alphabet de
      taille trois a une complexité de \(O_{\ndet}(2^n)\).
    }
  \end{block}
\end{frame}