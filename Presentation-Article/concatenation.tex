% !chktex-file 8
% !chktex-file 11
% !chktex-file 41

\section{Concaténation}

\setframetitle{La concaténation}

\subsection{Préambule}

\begin{frame}{\myframetitle}
  \begin{block}{La complexité exacte de la concaténation}
    Nous allons montrer que la concaténation a une complexité déterministe
    << exacte >> à \(m 2^n - 2^{n - 1}\).
  \end{block}
\end{frame}

\subsection{Preuve}

\begin{frame}{\myframetitle}
  \begin{block}{
      Définition des automes \(A\) et \(B\) (\(\mu = (m - n + 1) \bmod n\))
    }
    % LTeX: enabled=false
    \centering
    \begin{tikzpicture}[scale=0.5, transform shape]
      \node[state, initial] (0) {\(q_0\)};
      \node[state, right of=0] (1) {\(q_1\)};
      \node[state, draw=none, right of=1] (2) {\(\cdots\)};
      \node[state, right of=2] (3) {\(q_\mu\)};
      \node[state, draw=none, right of=3] (4) {\(\cdots\)};
      \node[state, right of=4] (5) {\(q_{m - 2}\)};
      \node[state, accepting, right of=5] (6) {\(q_{m - 1}\)};

      \node[
        draw,
        rectangle,
        thick,
        fit=(0) (1) (2) (3) (4) (5) (6),
        inner sep=0.5cm,
        minimum height=3cm
      ] (A) {};

      \node[scale=2] at ([shift={(-1cm,0cm)}] A.west) {\(A\)};

      \draw
      (0) edge[above] node{\(a,b\)} (1)
      (1) edge[above] node{\(a,b\)} (2)
      (2) edge[above] node{\(a,b\)} (3)
      (3) edge[above] node{\(a,b\)} (4)
      (4) edge[above] node{\(a,b\)} (5)
      (5) edge[above] node{\(a\)} (6)

      (5) edge[below, bend left=20] node{\(b\)} (0)

      (6) edge[above, bend left=20] node{\(b\)} (3)
      (6) edge[below, bend right=25] node{\(a\)} (0);

      \node[state, initial, below=5cm of 1] (0') {\(0\)};
      \node[state, right of=0'] (1') {\(1\)};
      \node[state, draw=none, right of=1'] (2') {\(\cdots\)};
      \node[state, right of=2'] (3') {\(n - 2\)};
      \node[state, accepting, right of=3'] (4') {\(n - 1\)};

      \node[
        draw,
        rectangle,
        thick,
        fit=(0') (1') (2') (3') (4'),
        inner sep=0.5cm,
        minimum height=2cm
      ] (B) {};

      \node[scale=2] at ([shift={(-1cm,0cm)}] B.west) {\(B\)};

      \draw
      (0') edge[above] node{\(a,b\)} (1')
      (1') edge[above] node{\(a,b\)} (2')
      (2') edge[above] node{\(a,b\)} (3')
      (3') edge[above] node{\(a,b\)} (4')

      (4') edge[above, loop above] node{\(a\)} (4')
      (4') edge[above, bend left=20] node{\(b\)} (0');
    \end{tikzpicture}
    % LTeX: enabled=true
  \end{block}
\end{frame}

\begin{frame}{\myframetitle}
  \begin{block}{Représentation de l'automate \(M\)}
    % LTeX: enabled=false
    \centering
    \begin{tikzpicture}[
        scale=0.7,
        transform shape,
        initial distance=1cm,
      ]
      \node[state, initial] (0) {\(q_0\)};
      \node[state, right=2.5cm of 0] (1) {\(q_{m - 2}\)};
      \node[state, right of=1] (2) {\(q_{m - 1}\)};

      \node[state, right of=2] (3) {\(0\)};
      \node[state, right of=3] (4) {\(1\)};
      \node[state, accepting, right=2.5cm of 4] (5) {\(n - 1\)};

      \draw
      (1) edge[above] node{\(a\)} (2)
      (3) edge[above] node{\(a,b\)} (4)
      (2) edge[above, bend left] node{\(a,b\)} (4)
      (5) edge[below, bend left] node{\(b\)} (3);

      \node[
        draw,
        rectangle,
        thick,
        fit=(0) (1) (2),
        inner sep=0.3cm
      ] (A) {};

      \node[
        draw,
        rectangle,
        thick,
        fit=(3) (4) (5) (5),
        inner sep=0.3cm,
        minimum height=3cm
      ] (B) {};

      \node[above of=A, scale=1.5] {\(A\)};
      \node[above of=B, scale=1.5] {\(B\)};
    \end{tikzpicture}
    % LTeX: enabled=true
  \end{block}
\end{frame}

\subsection{Conclusion}

\setframetitle{Conclusion}

\begin{frame}{\myframetitle}
  \begin{block}{Conclusion}
    Or, la concaténation de deux langages rationnels à une complexité
    déterministe \(m2^n - 2^{n - 1}\).\footnotemark{}
  \end{block}
  \footnotetext{\hyperlink{Yu1997}{\cite{Yu1997}}}
\end{frame}
