\section{Conclusion}

Durant cet article, nous avons défini deux transformations particulières
de langages réguliers~: le trognon d’un langage et le langage permuté. Pour
chacune d’elles, nous avons conçu un algorithme capable de construire, à
partir d’un automate non déterministe initial, un nouvel automate
reconnaissant le langage transformé.

\vphantom{}

Nous avons également fourni une preuve de correction pour chaque algorithme,
montrant que les automates produits reconnaissent exactement les langages
visés. Enfin, nous avons analysé la complexité en nombre d’états dans le pire
des cas.

\vphantom{}

Des perspectives intéressantes s’ouvrent à la suite de ce travail. En
particulier, affiner les bornes de complexité.
