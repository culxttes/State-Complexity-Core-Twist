\section{Le langage permuté}

Dans cette section, nous nous intéressons à un langage particulier, appelé
<< langage permuté >>. Nous commencerons par en donner une définition
formelle. Nous introduirons ensuite un algorithme de << twistage >> permettant
de construire un automate non déterministe reconnaissant ce langage, à partir
d’un automate initial. Nous établirirons ensuite la validité de cet algorithme
par une preuve, et analyserons sa complexité dans le pire des cas. Enfin, nous
conclurons cette section par une brève synthèse des résultats obtenus et des
perspectives de prolongement.

\subsection{Définition}

\begin{definition}[Le langage permuté]
  Le langage \(Twist(L)\) (aussi noté \(Permute(L)\)) est construit à partir
  de \(L\) en échangeant toutes les paires de lettres d'indice \(2k\) et
  \(2k + 1\) pour \(k \in \mathbb{N}\)~:
  \begin{gather*}
    \begin{align*}
      &Twist(\varepsilon) = \varepsilon,&
      &Twist(a) = a,&
      &Twist(a \cdot b \cdot v) = b \cdot a \cdot Twist(v),&
    \end{align*} \\
    \text{avec } (a, b) \in \Sigma^2 \text{ et } v \in \Sigma^*, \\
    Twist(L) = \bigcup_{w \in L} Twist(w).
  \end{gather*}
\end{definition}

\begin{example}
  \vspace{-\baselineskip}
  \begin{align*}
    L &= \{a, ab, aba, baba\}, \\
    Twist(L) &= \{a, ba, baa, abab\}.
  \end{align*}
\end{example}

Maintenant que nous savons ce qu’est le langage permuté, nous allons présenter
un algorithme permettant de construire un automate le reconnaissant, à partir
d’un automate reconnaissant le langage initial.

\subsection{Algorithme de twistage}

Notre algorithme prendra en entrée un automate non déterministe et renverra un
automate non déterministe. Mais, avant de rentrer dans le dur de l'algorithme,
nous regardons le principe de l'algorithme, puis nous verrons sa définition
formelle.

\subsubsection{Principe de l'algorithme}

Avant de twister un automate quelconque, on va s'intéresser à des cas
particuliers pour essayer de trouver une logique commune.

\vphantom{}

Le premier cas intéressant sont les automates qui forment qu'une longue
<< branche >>, c'est-à-dire que chaque état n'a qu'une transition vers un
autre état différent de lui-même et que l'automate ne possède qu'un seul état
initial~:

\begin{figure}[H]
  \centering
  \captionsetup{type=figure,justification=centering}
  \begin{tikzpicture}
    \tikzset{
      ->,
      >=stealth',
      node distance=2.5cm,
      every state/.style={thick},
      initial text=$ $,
    }
    \node[state, initial] (q1) {\(q_1\)};
    \node[state, right of=q1] (q2) {\(q_2\)};
    \node[state, accepting, right of=q2] (q3) {\(q_3\)};
    \node[state, right of=q3] (q4) {\(q_4\)};
    \node[state, right of=q4] (q5) {\(q_5\)};
    \node[state, accepting, right of=q5] (q6) {\(q_6\)};

    \draw
      (q1) edge[above] node{b} (q2)
      (q2) edge[above] node{a} (q3)
      (q3) edge[above] node{b} (q4)
      (q4) edge[above] node{a} (q5)
      (q5) edge[above] node{b} (q6);
  \end{tikzpicture}
  \caption{
    Exemple d'un automate qui forme qu'une << branche >>.
  }\label{fig:automate_branche}
\end{figure}

En effet, pour les automates de cette forme, il nous suffit d'échanger les
paires de transitions d'indice \(2k\) et \(2k + 1\) pour \(k \in\mathbb{N}\),
sauf pour la dernière transition si la branche est de longueur impaire~:

\begin{figure}[H]
  \centering
  \captionsetup{type=figure,justification=centering}
  \begin{tikzpicture}
    \tikzset{
      ->,
      >=stealth',
      node distance=2.5cm,
      every state/.style={thick},
      initial text=$ $,
    }
    \node[state, initial] (q1) {\(q_1\)};
    \node[state, right of=q1] (q2) {\(q_2\)};
    \node[state, accepting, right of=q2] (q3) {\(q_3\)};
    \node[state, right of=q3] (q4) {\(q_4\)};
    \node[state, right of=q4] (q5) {\(q_5\)};
    \node[state, accepting, right of=q5] (q6) {\(q_6\)};

    \draw
      (q1) edge[above] node{a} (q2)
      (q2) edge[above] node{b} (q3)
      (q3) edge[above] node{a} (q4)
      (q4) edge[above] node{b} (q5)
      (q5) edge[above] node{b} (q6);
  \end{tikzpicture}
  \caption{
    Exemple de l'automate figure~\ref{fig:automate_branche} après twistage.
  }
\end{figure}

De plus, les automates qui forment des arborescences dont les \emph{sections
interembranchements} sont de longueur paire et avec des longueurs de
\emph{sections terminales} quelconques, sont intéressants~:

\begin{figure}[H]
  \centering
  \captionsetup{type=figure,justification=centering}
  \begin{tikzpicture}
    \tikzset{
      ->,
      >=stealth',
      node distance=2.5cm,
      every state/.style={thick},
      initial text=$ $,
    }
    \node[state, initial] (q1) {\(q_1\)};
    \node[state, right of=q1] (q2) {\(q_2\)};
    \node[state, accepting, right of=q2] (q3) {\(q_3\)};
    \node[state, above right of=q3] (q4) {\(q_4\)};
    \node[state, right of=q4] (q5) {\(q_5\)};
    \node[state, accepting, right of=q5] (q6) {\(q_6\)};
    \node[state, right of=q3] (q7) {\(q_7\)};
    \node[state, accepting, right of=q7] (q8) {\(q_8\)};
    \node[state, right of=q8] (q9) {\(q_9\)};
    \node[state, accepting, below right of=q3] (q10) {\(q_{10}\)};
    \node[state, right of=q10] (q11) {\(q_{11}\)};
    \node[state, right of=q11] (q12) {\(q_{12}\)};

    \draw
      (q1) edge[above] node{b} (q2)
      (q2) edge[above] node{a} (q3)
      (q3) edge[above] node{b} (q4)
      (q4) edge[above] node{a} (q5)
      (q5) edge[above] node{b} (q6)
      (q3) edge[above] node{a} (q7)
      (q7) edge[above] node{b} (q8)
      (q8) edge[above] node{b} (q9)
      (q3) edge[above] node{b} (q10)
      (q10) edge[above] node{b} (q11)
      (q11) edge[above] node{a} (q12);
  \end{tikzpicture}
  \caption{
    Exemple d'un automate qui forme une arborescence dont les sections
    interembranchements sont de longueur paire.
  }\label{fig:automate_arbre}
\end{figure}

On devra alors appliquer la même logique que l'on a utilisée pour branches,
mais pour chaque section cette fois-ci~:

\begin{figure}[H]
  \centering
  \captionsetup{type=figure,justification=centering}
  \begin{tikzpicture}
    \tikzset{
      ->,
      >=stealth',
      node distance=2.5cm,
      every state/.style={thick},
      initial text=$ $,
    }
    \node[state, initial] (q1) {\(q_1\)};
    \node[state, right of=q1] (q2) {\(q_2\)};
    \node[state, accepting, right of=q2] (q3) {\(q_3\)};
    \node[state, above right of=q3] (q4) {\(q_4\)};
    \node[state, right of=q4] (q5) {\(q_5\)};
    \node[state, accepting, right of=q5] (q6) {\(q_6\)};
    \node[state, right of=q3] (q7) {\(q_7\)};
    \node[state, accepting, right of=q7] (q8) {\(q_8\)};
    \node[state, right of=q8] (q9) {\(q_9\)};
    \node[state, accepting, below right of=q3] (q10) {\(q_{10}\)};
    \node[state, right of=q10] (q11) {\(q_{11}\)};
    \node[state, right of=q11] (q12) {\(q_{12}\)};

    \draw
      (q1) edge[above] node{a} (q2)
      (q2) edge[above] node{b} (q3)
      (q3) edge[above] node{a} (q4)
      (q4) edge[above] node{b} (q5)
      (q5) edge[above] node{b} (q6)
      (q3) edge[above] node{b} (q7)
      (q7) edge[above] node{a} (q8)
      (q8) edge[above] node{b} (q9)
      (q3) edge[above] node{b} (q10)
      (q10) edge[above] node{b} (q11)
      (q11) edge[above] node{a} (q12);
  \end{tikzpicture}
  \caption{
    Exemple de l'automate figure~\ref{fig:automate_arbre} après twistage.
  }\
\end{figure}

Enfin, considérons les automates qui forment des arborescences dans lesquelles
les sections interembranchements de longueur impaire ne se séparent que par un
seul symbole. Dans ce cas, si la section suivante a une longueur paire, elle
doit également se séparer par un seul symbole. En revanche, si cette section
est de longueur impaire, alors aucune contrainte ne s'applique à elle~:

\begin{figure}[H]
  \centering
  \captionsetup{type=figure,justification=centering}
  \begin{tikzpicture}
    \tikzset{
      ->,
      >=stealth',
      node distance=2.5cm,
      every state/.style={thick},
      initial text=$ $,
    }
    \node[state, initial] (q1) {\(q_1\)};
    \node[state, right of=q1] (q2) {\(q_2\)};
    \node[state, above right of=q2] (q3) {\(q_3\)};
    \node[state, right of=q3] (q4) {\(q_4\)};
    \node[state, accepting, right of=q4] (q5) {\(q_5\)};
    \node[state, right of=q2] (q6) {\(q_6\)};
    \node[state, accepting, right of=q6] (q7) {\(q_7\)};
    \node[state, right of=q7] (q8) {\(q_8\)};

    \draw
      (q1) edge[above] node{b} (q2)
      (q2) edge[above] node{a} (q3)
      (q3) edge[above] node{b} (q4)
      (q4) edge[above] node{a} (q5)
      (q2) edge[above] node{a} (q6)
      (q6) edge[above] node{a} (q7)
      (q7) edge[above] node{b} (q8);
  \end{tikzpicture}
  \caption{
    Exemple d’un automate forme une arborescence avec des sections
    interembranchements contraintes.
  }\label{fig:automate_arbre2}
\end{figure}

On pourra ensuite appliquer la même logique qu'avant sans que l'automate
reconnaisse des mots qu'il ne devrait pas~:

\begin{figure}[H]
  \centering
  \captionsetup{type=figure,justification=centering}
  \begin{tikzpicture}
    \tikzset{
      ->,
      >=stealth',
      node distance=2.5cm,
      every state/.style={thick},
      initial text=$ $,
    }
    \node[state, initial] (q1) {\(q_1\)};
    \node[state, right of=q1] (q2) {\(q_2\)};
    \node[state, above right of=q2] (q3) {\(q_3\)};
    \node[state, right of=q3] (q4) {\(q_4\)};
    \node[state, accepting, right of=q4] (q5) {\(q_5\)};
    \node[state, right of=q2] (q6) {\(q_6\)};
    \node[state, accepting, right of=q6] (q7) {\(q_7\)};
    \node[state, right of=q7] (q8) {\(q_8\)};

    \draw
      (q1) edge[above] node{a} (q2)
      (q2) edge[above] node{b} (q3)
      (q3) edge[above] node{a} (q4)
      (q4) edge[above] node{b} (q5)
      (q2) edge[above] node{b} (q6)
      (q6) edge[above] node{b} (q7)
      (q7) edge[above] node{a} (q8);
  \end{tikzpicture}
  \caption{
    Exemple de l'automate figure~\ref{fig:automate_arbre2} après twistage.
  }
\end{figure}

Nous avons donc vu tous les automates sur lesquels il << suffit >> d'échanger
les paires de transitions pour que ça marche. Cependant, nous allons
maintenant voir un automate avec lequel il faut avoir une étape de transition
pour le transformer dans un cas qu'on vient d'observer.

\vphantom{}

Prenons maintenant un automate qui forme une arborescence, mais qui ne
respecte pas les conditions que nous avons fixées précédemment~:

\begin{figure}[H]
  \centering
  \captionsetup{type=figure,justification=centering}
  \begin{tikzpicture}
    \tikzset{
      ->,
      >=stealth',
      node distance=2.5cm,
      every state/.style={thick},
      initial text=$ $,
    }
    \node[state, initial] (q1) {\(q_1\)};
    \node[state, right of=q1] (q2) {\(q_2\)};
    \node[state, above right of=q2] (q3) {\(q_3\)};
    \node[state, right of=q3] (q4) {\(q_4\)};
    \node[state, accepting, right of=q4] (q5) {\(q_5\)};
    \node[state, right of=q2] (q6) {\(q_6\)};
    \node[state, accepting, right of=q6] (q7) {\(q_7\)};
    \node[state, right of=q7] (q8) {\(q_8\)};

    \draw
      (q1) edge[above] node{b} (q2)
      (q2) edge[above] node{a} (q3)
      (q3) edge[above] node{b} (q4)
      (q4) edge[above] node{a} (q5)
      (q2) edge[above] node{b} (q6)
      (q6) edge[above] node{a} (q7)
      (q7) edge[above] node{b} (q8);
  \end{tikzpicture}
  \caption{
    Exemple d’un automate forme une arborescence avec des sections
    interembranchements sans contraintes.
  }\label{fig:automate_arbre3}
\end{figure}

Dans ce cas, puisque l'automate de la figure~\ref{fig:automate_arbre3} ne
respecte pas les contraintes fixées précédemment (notamment sur la parité des
longueurs des sections interembranchements et leur séparation par un unique
symbole), on ne peut pas simplement échanger les paires de transitions comme
dans les exemples précédents.

\vphantom{}

Pour résoudre cela, transformons l’automate en dupliquant l’état
d’embranchement (ici \(q_2\)) autant de fois qu’il y a de transitions
sortantes avec des symboles différents. Dans cet exemple, \(q_2\) possède deux
transitions sortantes~: l'une étiquetée par \(a\) et l'autre par \(b\). On
créera donc deux copies distinctes de \(q_2\), chacune dédiée à un seul
symbole de transition. Cela permet d’isoler les chemins et de leur appliquer
ensuite les règles précédentes de transformation (comme l’échange des paires
de transitions).

\begin{figure}[H]
  \centering
  \captionsetup{type=figure,justification=centering}
  \begin{tikzpicture}
    \tikzset{
      ->,
      >=stealth',
      node distance=2.5cm,
      every state/.style={thick},
      initial text=$ $,
    }
    \node[state, initial] (q1) {\(q_1\)};
    \node[state, above right of=q1] (q2a) {\(q_{2a}\)};
    \node[state, right of=q1] (q2b) {\(q_{2b}\)};
    \node[state, right of=q2a] (q3) {\(q_3\)};
    \node[state, right of=q3] (q4) {\(q_4\)};
    \node[state, accepting, right of=q4] (q5) {\(q_5\)};
    \node[state, right of=q2] (q6) {\(q_6\)};
    \node[state, accepting, right of=q6] (q7) {\(q_7\)};
    \node[state, right of=q7] (q8) {\(q_8\)};

    \draw
      (q1) edge[above] node{b} (q2a)
      (q1) edge[above] node{b} (q2b)
      (q2a) edge[above] node{a} (q3)
      (q3) edge[above] node{b} (q4)
      (q4) edge[above] node{a} (q5)
      (q2b) edge[above] node{b} (q6)
      (q6) edge[above] node{a} (q7)
      (q7) edge[above] node{b} (q8);
  \end{tikzpicture}
  \caption{
    Exemple de l'automate figure~\ref{fig:automate_arbre3} après la
    duplication de \(q_2\).
  }\label{fig:automate_arbre4}
\end{figure}

Il nous reste donc à appliquer les précédentes règles de transformation~:

\begin{figure}[H]
  \centering
  \captionsetup{type=figure,justification=centering}
  \begin{tikzpicture}
    \tikzset{
      ->,
      >=stealth',
      node distance=2.5cm,
      every state/.style={thick},
      initial text=$ $,
    }
    \node[state, initial] (q1) {\(q_1\)};
    \node[state, above right of=q1] (q2a) {\(q_{2a}\)};
    \node[state, right of=q1] (q2b) {\(q_{2b}\)};
    \node[state, right of=q2a] (q3) {\(q_3\)};
    \node[state, right of=q3] (q4) {\(q_4\)};
    \node[state, accepting, right of=q4] (q5) {\(q_5\)};
    \node[state, right of=q2] (q6) {\(q_6\)};
    \node[state, accepting, right of=q6] (q7) {\(q_7\)};
    \node[state, right of=q7] (q8) {\(q_8\)};

    \draw
      (q1) edge[above] node{a} (q2a)
      (q1) edge[above] node{b} (q2b)
      (q2a) edge[above] node{b} (q3)
      (q3) edge[above] node{a} (q4)
      (q4) edge[above] node{b} (q5)
      (q2b) edge[above] node{b} (q6)
      (q6) edge[above] node{b} (q7)
      (q7) edge[above] node{a} (q8);
  \end{tikzpicture}
  \caption{
    Exemple de l'automate figure~\ref{fig:automate_arbre4} après le twistage.
  }
\end{figure}

Maintenant que nous savons transformer tous les automates qui forment des
arborescences, pour pouvoir transformer n'importe quel automate, nous allons
devoir introduire des << cycles >> (notions de théorie des graphes).

\vphantom{}

Lorsque des cycles sont présents, un même état peut être atteint par plusieurs
chemins différents, parfois de longueurs impaires. Cela complique
l'application des règles de transformation. Ainsi, chaque état intermédiaire
devra, comme nous l’avons fait précédemment avec \(q_2\), être dupliqué autant
de fois que nécessaire. Cette duplication permet d’isoler les différents
contextes d’accès à cet état, en fonction de la parité du chemin qui y mène,
et donc garantir que les règles de transformation restent applicables même en
présence de cycles.

\vphantom{}

Dans notre algorithme final, il ne sera pas nécessaire d’avoir une phase de
prétraitement séparée pour transformer l’automate. En effet, l’algorithme
intégrera directement la transformation (duplication des états) ainsi que le
twistage.

\subsubsection{Formalisation de l'algorithme}

Maintenant que nous avons l'idée de l'algorithme, nous verrons sa
définition formelle~:
\begin{minted}{text}
twister(M):
  N |\(\leftarrow\)| (|\(\varnothing\)|, |\(\varnothing\)|, |\(\varnothing\)|, |\(\varnothing \to \varnothing\)|)
  |\textbf{for}| q |\textbf{in}| N.I |\textbf{do}|
    N.Q |\(\leftarrow\)| |\(\texttt{N.Q} \cup \texttt{\{\,q\,\}}\)|
    N.I |\(\leftarrow\)| |\(\texttt{N.I} \cup \texttt{\{\,q\,\}}\)|
    N.F |\(\leftarrow\)| |\(\texttt{N.F} \cup \texttt{(N.F} \cap \texttt{\{\,q\,\})}\)|
    twister_aux(M, q, N)
  |\textbf{return}| N
\end{minted}

\newpage
\begin{minted}{text}
twister_aux(M, q, N):
  |\textbf{for}| (p, a) |\textbf{in}| N.|\(\delta\texttt{(q)}\)| |\textbf{do}|
    |\textbf{if}| p |\(\in\)| M.F |\textbf{then}|
      N.Q |\(\leftarrow\)| |\(\texttt{N.Q} \cup \texttt{\{\,p}_\texttt{a} \texttt{\,\}}\)|
      N.F |\(\leftarrow\)| |\(\texttt{N.F} \cup \texttt{\{\,p}_\texttt{a} \texttt{\,\}}\)|
      N.|\(\delta\)| |\(\leftarrow\)| |\(\texttt{N.}\delta \cup \texttt{\{\,(q, a, } \texttt{p}_\texttt{a}\texttt{)\,\}}\)|
    |\textbf{for}| (r, b) |\textbf{in}| N.|\(\delta\texttt{(p)}\)| |\textbf{do}|
      A |\(\leftarrow\)| N
      A.Q |\(\leftarrow\)| |\(\texttt{A.Q} \cup \texttt{\{\,} \texttt{p}_\texttt{b}, \texttt{r\,\}}\)|
      A.F |\(\leftarrow\)| |\(\texttt{A.F} \cup \texttt{(M.F} \cap \texttt{\{\,r\,\})}\)|
      A.|\(\delta\)| |\(\leftarrow\)| |\(\texttt{A.}\delta \cup \texttt{\{\,(q, b, } \texttt{p}_\texttt{b}\texttt{)}, \texttt{(} \texttt{p}_\texttt{b} \texttt{, a, r)\,\}}\)|
      |\textbf{if}| r |\(\notin\)| N.Q |\textbf{then}|
        A |\(\leftarrow\)| twister_aux(M, r, A)
      N |\(\leftarrow\)| A
  |\textbf{return}| N
\end{minted}

\begin{remark}
  Cet algorithme parcourt les paires de transitions sortantes d’un état donné,
  puis duplique l’état intermédiaire en fonction du symbole de transition
  utilisé. Il applique ensuite l’opération de twistage en inversant l’ordre
  des deux transitions, avant de poursuivre récursivement le traitement sur
  l'état d’arrivée de la paire de transitions.
\end{remark}

\subsection{Preuve de notre algorithme}

Pour prouver que notre algorithme est correct et construit bien l'automate
du langage permuté, nous allons d'abord montrer que tous mots reconnus
par l'automate sont des mots appartenant au langage permuté. Enfin, nous
allons montrer que tous les mots du langage permuté sont reconnus par 
l'automate.

\subsubsection{Tous les mots reconnus sont des mots du langage permut\'e}

\begin{proof}\label{proof:perm_rec}
  Soit \(M \in NFA(\Sigma, \eta)\), \(N = \texttt{twister}(M)\) et un mot
  \(w\) reconnu par l'automate \(N\). Pour prouver qu'il fait partie du
  langage permuté, nous allons faire une preuve par récurrence sur la longueur
  de ce mot~:

  \vphantom{}

  \begin{itemize}
    \item[\bullet~\textbf{Initialisation~:}] \(\lvert w \rvert \leq 1\)
      \begin{itemize}
        \item[\circ] Si \(\lvert w \rvert = 0\), c'est qu'alors qu'un des
          états initiaux est finaux aussi, or dans l'algorithme, on duplique
          les états initiaux (en les laissant finaux s'ils le sont) donc cela
          veut dire que \(M\) reconnait lui aussi \(\varepsilon\). Ainsi,
          \(w \in Twist(L(M))\).

        \vphantom{}

        \item[\circ] Si \(\lvert w \rvert = 1\), c'est qu'alors après en une
          transition d'un état initial, il y a un état final. Or dans
          l'algorithme, s'il y a une transition qui amène vers un état final,
          alors cette transition est gardée et l'état reste final. Par
          conséquent, cela veut dire que l'automate \(M\) possède aussi cette
          transition et le permuté d'un mot composé d'un symbole est-ce même
          mot. Ainsi, \(w \in Twist(L(M))\).
      \end{itemize}

      \vphantom{}

    \item[\bullet~\textbf{Hérédité~:}] \(\lvert w \rvert \geq 2\)

      On peut donc décomposer le mot \(w\) tel que \(w = a \cdot b \cdot u\),
      avec \((a, b) \in \Sigma^2\) et \(u \in \Sigma^*\). S'il est reconnu,
      cela veut dire qu'il existe un chemin partant d'un état initial \(i\)
      avec une transition en \(a\) puis en \(b\). Or dans notre algorithme,
      nous inversons les paires de transitions en partant des états initiaux,
      cela veut dire que dans l'automate \(M\) il existe une transition
      partant du même état initial \(i\) en \(b\) ensuite en \(a\). Il reste
      donc à savoir si le mot \(u\) fait partie du langage permuté de
      l'automate \(M\) qu'aurait comme états initiaux \(\delta(i, b
      \cdot a)\). Or dans notre algorithme, nous faisons un appel récursif
      pour calculer l'automate twister en partant de \(i\), ainsi par
      hypothèse de récurrence \(w \in Twist(L(M))\).
  \end{itemize}
\end{proof}

\subsubsection{Tous les mots du langage permuté sont reconnus}

\begin{proof}[Esquisse de preuve]
  Nous procédons également par récurrence sur la longueur des mots du langage
  de base \(L(M)\). L’objectif est de montrer que pour tout mot \(w \in
  L(M)\), sa permutation correspondante (par inversion de paires de
  transitions) est bien reconnue par l’automate \(N = \texttt{twister}(M)\).

  \vphantom{}

  \begin{itemize}
    \item[\bullet~\textbf{Initialisation~:}] \(\lvert w \rvert \leq 1\)

      Alors \(w\) est soit le mot vide, soit un mot d’un seul symbole. Dans
      les deux cas, comme vu précédemment, l’algorithme conserve ces cas en
      dupliquant les états initiaux et finaux sans modification, ce qui
      garantit que \(Twist(w) \in L(N)\).

    \vphantom{}

    \item[\bullet~\textbf{Hérédité~:}]  \(\lvert w \rvert \geq 2\)
      
      Soit \(w = b \cdot a \cdot u \in L(M)\), d'après la construction de
      l’algorithme, les transitions sont inversées par paire. Ainsi, si
      \(M\) reconnaît le préfixe \(b \cdot a\), alors l’automate \(N\)
      reconnaît le préfixe \(a \cdot b\). Par hypothèse de récurrence, la
      mot \(Twist(u)\) est également reconnue par \(N\), ce qui conclut que
      \(Twist(w) \in L(N)\).
  \end{itemize}
\end{proof}

\subsection{Pire cas de notre algorithme}

Dans cette section, nous établirons une borne supérieure sur le nombre
d’états de l’automate que notre algorithme peut engendrer. Cette estimation,
distincte de la complexité du langage permuté, nous offre néanmoins un
plafond pour la complexité en états.

\vphantom{}

Dans notre pire cas, notre algorithme dupliquera tous les états par au pire le
nombre de symboles de l’alphabet plus un (puisqu'on a les états indicés et
ceux non indicés). Par conséquent, dans le pire des cas, l’automate produit
aura \(n(\lvert \Sigma \rvert + 1)\) états, avec n le nombre d’états de
l’automate de départ.

\subsection{Conclusion}

Pour conclure, nous venons de créer un algorithme qui prend n’importe quel
automate en entrée et qui calcule l’automate du langage permuté de cet
automate. De plus, notre algorithme crée un automate dans le pire des cas à
\(n(\lvert \Sigma \rvert + 1)\). états. Par conséquent, la complexité en états
du twistage d’un langage est d’au moins \(n(\lvert \Sigma \rvert + 1)\).

\vphantom{}

Malgré la petite complexité, on suppose que cette complexité n'est pas
éteinte. Néanmoins, par manque de temps, aucune preuve de cette hypothèse ne
sera pas fournie. Il serait donc intéressant d'étudier la complexité en état
du twistage plus en détail dans un autre article.
