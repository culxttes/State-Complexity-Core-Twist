\section{Préliminaires}

Dans cette section préliminaire, nous allons rappeler les notions
fondamentales nécessaires à la compréhension du reste du document. Nous y
définirons notamment les concepts d’alphabet, de mot, de langage, d’automate
et de complexité en nombre d’états. Ces définitions seront accompagnées
d’exemples et de remarques pour en faciliter l’assimilation

% ----------------------------------------------------------------------------
\subsection{Les mots}

\begin{definition}[Alphabet]
  Un \textit{alphabet} \(\Sigma\) est un ensemble fini non-vide de
  \textit{symboles}.
\end{definition}

\begin{definition}[Mot]
  Un \textit{mot} est une suite finie de symboles sur un alphabet \(\Sigma\).
  Le mot composé de zéro symbole est appelé mot vide et est noté 
  \(\varepsilon\). De plus, l'ensemble des mots sera noté \(\Sigma^*\)
\end{definition}

\begin{example}
  \vspace{-\baselineskip}
  \begin{align*}
    \Sigma &= \{a, b, c, d\}, \\
    w &= abbcdda.
  \end{align*}
\end{example}

\begin{definition}[Longueur d'un mot]
  On parlera de la \textit{longueur d'un mot} \(w\) noté \(\lvert w \rvert\)
  pour désigner le nombre de symboles qui le composent. 
\end{definition}

\begin{example}
  \vspace{-\baselineskip}
  \begin{align*}
    w &= abbcdda, \\
    \lvert w \rvert &= 7.
  \end{align*}
\end{example}

\begin{definition}[Concaténation de deux mots]
  On notera la \textit{concaténation} de deux mots \(u = a_0 \cdots a_n\) et 
  \(v = b_0 \cdots b_m\) par \(u \cdot v\). Qui est ainsi égal à \(u \cdot v =
  a_0 \cdots a_n b_0 \cdots b_m\). On notera que~:

  \vphantom{}

  \begin{itemize}
    \item[\bullet] La concaténation est associative \((w \cdot u) \cdot v = w
      \cdot (u \cdot v)\).
    \item[\bullet] La concaténation admet un élément neutre \(u \cdot 
      \varepsilon = \varepsilon \cdot u = u\).
  \end{itemize}
\end{definition}

\begin{example}
  \vspace{-\baselineskip}
  \begin{align*}
    u &= aba, \\
    v &= cdcd, \\
    u \cdot v &= abacdcd.
  \end{align*}
\end{example}

\begin{definition}[Facteur, préfixe, suffixe d'un mot]
  Un \textit{facteur} \(u\) d'un mot \(w\) est une suite extraite de la suite
  de lettre qui composent le mot \(w\)~:
  \begin{align*}
    u \text{ est un facteur de } w \Longleftrightarrow \exists (v, x) \in
    (\Sigma ^ *)^2 ~|~ w = v \cdot u \cdot x.
  \end{align*} 

  \begin{itemize}
    \item[\bullet] Par ailleurs, on parlera de \textit{préfixe} quand \(v = 
      \varepsilon\).
    \item[\bullet] Enfin, on parlera de \textit{suffixe} quand \(x =
      \varepsilon\).
  \end{itemize}

  \vphantom{}

  Par ailleurs, on remarquera que \(\varepsilon\) est~: \textit{préfixe}, 
  \textit{suffixe} et \textit{facteur} de tout mot.
\end{definition}

\begin{example}
  Dans cet exemple, on a que \(u\) est un facteur de \(w\), que \(v\) est un
  préfixe de \(w\) et enfin que \(y\) est un suffixe de \(w\)~:
  \begin{align*}
    w &= abbcdda, \\
    u &= bcd, \\
    v &= abb, \\
    y &= dda.
  \end{align*}
\end{example}

\begin{definition}[Le miroir d'un mot]
  Le \textit{miroir} d'un mot \(w\) noté \(\overleftarrow{w}\). Qui est ainsi
  défini récursivement comme ceci~:
  \begin{align*}
    \overleftarrow{\varepsilon} &= \varepsilon, \\
    \overleftarrow{u \cdot a} &= a \cdot \overleftarrow{u}, \\
    \text{Avec } a \in~&\Sigma \text{ et } u \in \Sigma^*.
  \end{align*}
\end{definition}

\begin{example}
  \vspace{-\baselineskip}
  \begin{align*}
    w &= abbcdda, \\
    \overleftarrow{w} &= addcbba.
  \end{align*}
\end{example}

% ----------------------------------------------------------------------------
\subsection{Les langages}

\begin{definition}[Langage]
  Un \textit{langage} \(L\) est un ensemble de mots sur un alphabet fini
  \(\Sigma\). On appellera \textit{langage vide} le langage ne comportant
  aucun mot et sera noté \(\varnothing\).
\end{definition}

\begin{example}
  \vspace{-\baselineskip}
  \begin{align*}
    \Sigma &= \{a, b, c, d\}, \\
    L_1 &= \{a, aa, bc, da, \varepsilon\}, \\
    L_2 &= \varnothing.
  \end{align*}
\end{example}

\begin{definition}[L'union de langages]
  L'\textit{union} de deux langages sera notée \(L_1 \cup L_2\) et est définie
  comme ceci~:
  \begin{gather*}
      L_1 \cup L_2 = \{w \in \Sigma ^ * \mid w \in L_1 \lor w \in L_2\}.
  \end{gather*}
\end{definition}

\begin{example}
  \vspace{-\baselineskip}
  \begin{align*}
    L_1 &= \{\varepsilon, a, aa, bc, da\}, \\
    L_2 &= \{d, aa, cd\}, \\
    L_1 \cup L_2 &= \{\varepsilon, a, d, aa, cd, bc, da\}. \\
  \end{align*}
\end{example}

\begin{definition}[Concaténation de langages]
  La \textit{concaténation} de deux langages sera notée \(L_1 \cdot L_2\) et
  est définie grâce à la concaténation des mots qui composent les langages~:
  \begin{gather*}
    L_1 \cdot L_2 = \{u \cdot v ~|~ u \in L_1, v \in L_2\}.
  \end{gather*}
\end{definition}

\begin{example}
  \vspace{-\baselineskip}
  \begin{align*}
    L_1 &= \{\varepsilon, a, aa\}, \\
    L_2 &= \{d, cc\}, \\
    L_1 \cdot L_2 &= \{d, ad, aad, cc, acc, aacc\}.
  \end{align*}
\end{example}

\begin{definition}[Copie n-ième d'un langage]
  La \textit{copie n-ième} d'un langage \(L\) notée \(L^n\) et est définie
  récursivement comme ceci~:
  \begin{align*}
    L^0 &= \{\varepsilon\}, \\
    L^n &= L^{n - 1} \cdot L.
  \end{align*}
\end{definition}

\begin{example}
  \vspace{-\baselineskip}
  \begin{align*}
    L &= \{\varepsilon, a\}, \\
    L^3 &= \{\varepsilon, a, aa, aaa\}.
  \end{align*}
\end{example}

\begin{definition}[L'étoile (de \textsc{Kleene}) d'un langage]
  L'\textit{étoile} (de \textsc{Kleene}) d'un langage noté \(L^*\), sera
  définie comme ceci~:
  \begin{gather*}
    L^* = \bigcup_{i \geq 0} L^i.
  \end{gather*}
\end{definition}

\begin{example}
  \vspace{-\baselineskip}
  \begin{align*}
    L &= \{a\}, \\
    L^* &= \{\varepsilon\} \cup \{a\} \cup \{aa\} \cup \cdots
  \end{align*}
\end{example}

\begin{remark}
  Grâce à cette opération, on comprend dorénavant la notion \(\Sigma^*\) comme
  l'ensemble des mots de l'alphabet \(\Sigma\), il pourra donc aussi être vu
  comme le langage contenant tous les mots.
\end{remark}

\begin{definition}[Langages rationnel]
  On note \(Rat(\Sigma^*)\) le plus petit ensemble de langages sur
  \(\Sigma^*\) vérifiant les propriétés suivantes~:

  \vphantom{}

  \begin{itemize}
    \item[\bullet] \(\varnothing \in Rat(\Sigma^*)\).
    \item[\bullet] \(\{a\} \in Rat(\Sigma^*) \text{ avec } a \in \Sigma\).
    \item[\bullet] \(Rat(\Sigma^*)\) est fermé pour les opérations d'union,
    de concaténation et pour l'étoile.
  \end{itemize}
\end{definition}

\begin{remark}
  Pour simplifier l’écriture, nous écrirons \(+\) pour représenter l’union, et
  nous utiliserons également \(ab\) au lieu de \(\{ab\}\).
\end{remark}

\begin{example}
  \vspace{-\baselineskip}
  \begin{align*}
    L = \varnothing + (a^* \cdot (b + \varepsilon)).
  \end{align*}
\end{example}

% ----------------------------------------------------------------------------
\subsection{Les automates}

Nous allons parler des automates sans \(\varepsilon\)-transition (des 
automates utilisent des \\\(\varepsilon\)-transitions, comme ceux de 
\textsc{Thompson} \cite{thompson1968programming}, utilisés par nos 
ordinateurs). Pour autant, les automates que nous verrons ne sont pas limités
par le manque de ces transitions.

\begin{definition}[Automate]
  Un \textit{automate} est un objet mathématique \textit{reconnaissant} un
  langage. On notera \(M \in AFN(\Sigma, \eta)\) l'automate qui a pour
  \textit{transition} des valeurs dans \(\Sigma\), des valeurs d'\textit{état}
  dans \(\eta\).

  \vphantom{}

  On aura aussi, \(AFN(\Sigma, \eta)\), l'ensemble des automates finis 
  \textit{non déterministes} de valeur de transition dans \(\Sigma\) et de 
  valeur d'état dans \(\eta\). Enfin, on écrira \(L(M)\) pour désigner le
  langage qu'il reconnait.

  \vphantom{}

  Un automate est un tuple qu'on peut écrire de cette forme \(M = (Q, I, F, 
  \delta)\) avec~:
  \begin{alignat*}{3}
    &\cmath{Q} && \subseteq \eta &&\text{États de l'automate.} \\
    &\cmath{I} && \subseteq Q &&\text{États initiaux.} \\
    &\cmath{F} && \subseteq Q &&\text{États finaux.} \\
    &\cmath{\delta}&&: Q \times \Sigma \to \mathcal{P}(Q)~ &&\text{Fonction de
    transition.}
  \end{alignat*}
\end{definition}

\begin{remark}[Extension de la fonction de transition]
  On peut étendre la fonction de transition pour ce qu’elle prenne en entrée
  non plus un simple symbole, mais un mot. Sa signature devient alors 
  \(\delta~: Q \times \Sigma^* \to \mathcal{P}(Q)\), et elle est définie comme
  suit~:
  \begin{align*}
    \delta(q, \varepsilon) &= \{q\}, \\
    \delta(q, a \cdot w) &= \bigcup_{p \in \delta(q, a)} \delta(p, w),
    \text{ avec } a \in \Sigma.
  \end{align*}
\end{remark}

\begin{definition}[Reconnaissance d'un mot par un automate]
  Un mot est \textit{reconnu} (ou \textit{accepté}) s'il existe une suite de
  transition partant d'un état initial vers un état final~:
  \begin{align*}
    w \text{ est accepté par } M \Longleftrightarrow \left( \bigcup_{i \in I}
    \delta(i, w) \right) \cap F \neq \varnothing.
  \end{align*}
\end{definition}

\begin{definition}[Langage d'un automate]
  Le \textit{langage d'un automate} est donc l'ensemble des mots qu'il
  reconnait~:
  \begin{align*}
    L(M) = \{w \in \Sigma^* \mid w \text{ est accepté par } M\}.
  \end{align*}
\end{definition}

\begin{example}[Représentation d'un automate]
  Un automate peut se représenter à l'aide d'un graphe orienté et valué
  particulier. Par exemple, si on veut représenter \(M = (\{q_1, q_2, q_3,
  q_4, q_5\}, \{q_1, q_2\},\{q_2, q_3\}, \delta)\) avec \(M \in AFN(\Sigma,
  \eta)\), \(\Sigma = \{a, b\}\), \(\eta = \{q_1, q_2, q_3, q_4, q_5\}\) et
  \(\delta\) défini comme ceci~:
  \begin{align*}
    \delta(q_1, a) &= \{q_2, q_4\}, & \delta(q_3, b) & = \{q_4\}, \\
    \delta(q_1, b) &= \varnothing,  & \delta(q_4, a) & = \{q_5\}, \\
    \delta(q_2, a) &= \varnothing,  & \delta(q_4, b) & = \{q_3\}, \\
    \delta(q_2, b) &= \varnothing,  & \delta(q_5, a) & = \{q_4\}, \\
    \delta(q_3, a) &= \{q_3\},      & \delta(q_5, b) & = \{q_5\}.
  \end{align*}
  \begin{figure}[H]
    \centering
    \captionsetup{type=figure,justification=centering}
    \begin{tikzpicture}
      \tikzset{
        ->,
        >=stealth',
        node distance=3cm,
        every state/.style={thick},
        initial text=$ $,
      }
      \node[state, initial] (q1) {\(q_1\)};
      \node[state, initial right, accepting, right of=q1] (q2) {\(q_2\)};
      \node[state, above of=q2] (q4) {\(q_4\)};
      \node[state, accepting, right of=q2] (q3) {\(q_3\)};
      \node[state, right of=q4] (q5) {\(q_5\)};

      \draw
        (q1) edge[above] node{a} (q4)
        (q1) edge[below] node{a} (q2)
        (q4) edge[bend right, below] node{b} (q3)
        (q4) edge[above] node{a} (q5)
        (q3) edge[above] node{b} (q4)
        (q3) edge[loop right] node{a} (q3)
        (q5) edge[bend right, above] node{b} (q4)
        (q5) edge[loop right] node{b} (q5);
    \end{tikzpicture}
    \caption{Exemple de représentation graphique d'un automate.}
  \end{figure}

  On peut voir que les états initiaux ont une petite flèche qui pointe sur eux
  et que les états finaux ont un double contour. Et, que les transitions sont
  symbolisées par des flèches entre les états et que ces flèches sont 
  labellisées.
\end{example}

\begin{definition}[Automate déterministe]
  Un automate est dit \textit{déterministe} quand tous ses états vont au
  maximum à un état par symbole et que l'automate ne possède qu'un seul état
  initial.
  \begin{gather*}
    \lvert I \rvert = 1 \land \forall q \in Q, \forall a \in \Sigma, \lvert 
    \delta(q, a) \rvert \leq 1 \Longleftrightarrow M \in DFA(\Sigma, \eta)
  \end{gather*}
  avec \(DFA(\Sigma, \eta)\) l'ensemble des automates déterministe.
\end{definition}

\begin{example}[Représentation d'un automate déterministe]
  \vspace{-\baselineskip}
  \begin{figure}[H]
    \centering
    \captionsetup{type=figure,justification=centering}
    \begin{tikzpicture}
      \tikzset{
        ->,
        >=stealth',
        node distance=3cm,
        every state/.style={thick},
        initial text=$ $,
      }
      \node[state, initial] (q1) {\(q_1\)};
      \node[state, accepting, right of=q1] (q2) {\(q_2\)};
      \node[state, right of=q2] (q3) {\(q_3\)};

      \draw
        (q1) edge[above] node{b} (q2)
        (q2) edge[above] node{a} (q3)
        (q3) edge[bend right, above] node{a} (q2)
        (q3) edge[loop right] node{b} (q3);
    \end{tikzpicture}
    \caption{Exemple de représentation graphique d'un automate déterministe.}
  \end{figure}
\end{example}

\begin{definition}[L'ensemble des langages reconnaissable]
  On note \(Rec(\Sigma^*)\), l'ensemble des langages de \(\Sigma^*\) dit 
  \textit{reconnaissable}~:
  \begin{align*}
    L \in Rec(\Sigma^*) \Longleftrightarrow \exists M \in NFA(\Sigma, \eta)
    \land L = L(M).
  \end{align*}
\end{definition}

\begin{theorem}[Théorème de \textsc{Kleene}]
  Le mathématicien \textsc{Stephen Cole Kleene} a démontré que~:
  \begin{align*}
    Rec(\Sigma^*) = Rat(\Sigma^*).
  \end{align*}
  On pourra alors passer de langage rationnel à automate et vice-versa.
\end{theorem}

\begin{remark}
  La notation \(\lvert M \rvert\) représentera le nombre d'états de l'automate
  \(M\).
\end{remark}

\begin{definition}[Automate minimal]
  Un automate est dit \textit{minimal} lorsqu’il s’agit de l’automate ayant le
  plus petit nombre d’états possible qui reconnaît un langage donné~:
  \begin{align*}
    M \text{ est minimal} \Longrightarrow \nexists N \in NFA(\Sigma, \eta)
    \land L(M) = L(N).
  \end{align*}
  On parlera d'automate \text{déterministe minimal} quand~:
    \begin{align*}
    M \text{ est déterministe minimal} \Longrightarrow \nexists N \in 
    DFA(\Sigma, \eta) \land L(M) = L(N).
  \end{align*}
\end{definition}

\begin{example}[Représentation d'un automate minimal]
  Dans cet exemple, l'automate \(M_2\) est bien plus petit que l'automate
  \(M_1\), c'est même le plus petit automate reconnaissant leur langage
  (\(a^*\))
  \begin{figure}[H]
    \centering
    \captionsetup{type=figure,justification=centering}
    \begin{tikzpicture}
      \tikzset{
        ->,
        >=stealth',
        node distance=3cm,
        every state/.style={thick},
        initial text=$ $,
      }
      \node[state, initial, accepting] (q1) {\(q_1\)};
      \node[left=1cm of q1] {\(M_1\)};
      \node[state, accepting, right of=q1] (q2) {\(q_2\)};
      \node[state, accepting, right of=q2] (q3) {\(q_3\)};
      \node[state, initial, accepting, below=1cm of q1] (q4) {\(q_1\)};
      \node[left=1cm of q4] {\(M_2\)};

      \draw
        (q1) edge[above] node{a} (q2)
        (q2) edge[above] node{a} (q3)
        (q3) edge[loop right] node{a} (q3)
        (q4) edge[loop right] node{a} (q4);
    \end{tikzpicture}
    \caption{
      Exemple de représentation graphique d'un automate minimal.
    }\label{fig:ex_minimal}
  \end{figure}
\end{example}

%-----------------------------------------------------------------------------
\subsection{La complexité en état}

\begin{definition}[Complexité en état d'un langage (rationnel)]
  La \textit{complexité en état} d'un langage est le nombre d'états de
  l'automate minimal reconnaissant ce langage~:
  \begin{gather*}
    \mathcal{C}(L) = \lvert M \rvert \land L(M) = L \text{ avec} \\ 
    M \text{ est un automate minimal.}
  \end{gather*}
  Cette complexité se scinde en deux parties~:

  \vphantom{}

  \begin{itemize}
    \item[\bullet] La complexité non déterministe (\(\mathcal{C}_{ndet}\))
    quand~:
    \begin{align*}
      M \in NFA(\Sigma, \eta).
    \end{align*}
    \item[\bullet] La complexité déterministe (\(\mathcal{C}_{det}\))
    lorsque~:
    \begin{align*}
      M \in DFA(\Sigma, \eta).
    \end{align*}
  \end{itemize}
\end{definition}

\begin{example}
  Si on reprend le langage de la figure~\ref{fig:ex_minimal}, \(M_2\) est
  l'automate minimal reconnaissant \(a^*\) et il n'a qu'un état, de ce fait ce
  langage à une complexité en état déterministe de \(1\). Et, donc aussi, une
  complexité non déterministe de \(1\), puisque tout automate déterministe est
  dans \(NFA(\Sigma, \eta)\) et qu'on ne peut pas faire un automate à moins
  d'un état.
\end{example}

\begin{definition}[Complexité en état opérationnel]
  La \textit{complexité en état opérationnel} peut être vu comme la borne
  supérieure de la complexité en état du langage produit par l'opération.
  Autrement dit, \(f\) a une complexité de \(g(n_1, \ldots, n_k)\) quand~:
  \begin{align*}
    \forall (L_1, \ldots, L_k) \in \mathcal{P}(\Sigma^*) \quad
    \mathcal{C}(f(L_1, \ldots, L_k)) \leq 
    g(\mathcal{C}(L_1), \ldots, \mathcal{C}(L_k)).
  \end{align*}
  De même que la complexité en état, la complexité en état opérationnel se
  scinde en deux parties~:

  \vphantom{}

  \begin{itemize}
    \item[\bullet] La complexité non déterministe quand~:
    \begin{align*}
      \mathcal{C}_{ndet}(f(L_1, \ldots, L_k)) \leq 
      g(\mathcal{C}_{ndet}(L_1), \ldots, \mathcal{C}_{ndet}(L_k)).
    \end{align*}
    \item[\bullet] La complexité déterministe lorsque~:
    \begin{align*}
      \mathcal{C}_{det}(f(L_1, \ldots, L_k)) \leq 
      g(\mathcal{C}_{det}(L_1), \ldots, \mathcal{C}_{det}(L_k)).
    \end{align*}
  \end{itemize}
\end{definition}

\begin{remark}
  La complexité en état déterministe sera toujours supérieure ou égale à la
  complexité en état non déterministe, dû au fait que les automates
  déterministes sont aussi des automates non déterministe, mais avec des
  contraintes en plus.
\end{remark}

\begin{example}
  Voici quelques complexités en état opérationnel connues et importantes~:
  \begin{table}[H]
    \centering
    \captionsetup{type=table,justification=centering}
    \renewcommand{\arraystretch}{2}
    \begin{tabular}{|c|c|c|}
    \hline
      \textbf{Fonction} & \textbf{Complexité} & \textbf{Source} \\
      \hline
      L'union non déterministe & \(m + n\) & \textsc{Holzer} et 
      \textsc{Kutrib}~\cite{holzer_kutrib} \\
      \hline
      L'union déterministe & \(m n\) & 
      \textsc{Maslov}~\cite{maslov1970estimates} \\
      \hline
      L'intersection non déterministe & \(m n\) & \textsc{Holzer} et 
      \textsc{Kutrib}~\cite{holzer_kutrib} \\
      \hline
      L'intersection déterministe & \(m n\) & 
      \textsc{Maslov}~\cite{maslov1970estimates} \\
      \hline
      La concaténation non déterministe & \(m + n\) & \textsc{Holzer} et 
      \textsc{Kutrib}~\cite{holzer_kutrib} \\
      \hline
      La concaténation déterministe & \(m 2^n - 2^{n - 1}\) & 
      \textsc{Maslov}~\cite{maslov1970estimates} \\
      \hline
      L'étoile (de \textsc{Kleene}) non déterministe & \(n + 1\) &
      \textsc{Holzer} et \textsc{Kutrib}~\cite{holzer_kutrib} \\
      \hline
      L'étoile (de \textsc{Kleene}) déterministe & \(\frac{3}{4}2^n\) & 
      \textsc{Maslov}~\cite{maslov1970estimates} \\
      \hline
    \end{tabular}
    \caption{
      Tableau de complexité en état opérationnel de plusieurs opérations.
    }
  \end{table}
\end{example}

\begin{remark}[Complexité de la déterminisation]
  L'opération de \textit{déterminisation}, c'est-à-dire l'opération de rendre
  un automate non déterministe quelconque en un automate déterministe à une
  complexité en état de \(2^n\).

  Même si cette opération ne rentre pas réellement dans notre définition de la
  complexité en état, car prenant en entrée un automate non déterministe et
  renvoyant un automate déterministe.
\end{remark}
