\section{Introduction}

Cet article s'intéresse à la complexité en nombre d'états des automates
associés à des langages rationnels soumis à des transformations spécifiques.
Plus précisément, nous étudions deux opérations définies sur les langages
rationnels : le << trognon >> et le << langage permuté >>.

\vphantom{}

La première transformation, appelée << trognon >>, consiste à retirer, dans
chaque mot du langage de départ, un préfixe dont le miroir est également un
suffixe. Le mot central ainsi isolé constitue le << trognon >> du mot. Nous
proposons un algorithme, que nous appelons algorithme de << grignotage >>,
permettant, à partir d’un automate non déterministe reconnaissant un langage
rationnel donné, de construire un automate reconnaissant le trognon de ce
langage. Nous détaillons le principe de l’algorithme, formalisons sa
construction, en prouvons la correction et analysons sa complexité dans le
pire des cas.

\vphantom{}

La seconde transformation, appelée << langage permuté >>, consiste à permuter
toutes les paires de lettres consécutives situées à des positions paires et
impaires dans les mots du langage (positions \(2k\) et \(2k + 1\)). Pour cette
transformation également, nous construisons un algorithme, que nous nommons
algorithme de << twistage >>, qui transforme un automate reconnaissant le
langage de départ en un automate reconnaissant le langage permuté. Nous
décrivons le fonctionnement de cet algorithme, en démontrons la validité et
étudions sa complexité.

\vphantom{}

L’objectif principal de ce travail est d’évaluer l’impact de ces opérations
sur la taille des automates résultants, en particulier dans le cadre non
déterministe. Pour chaque transformation, nous établissons des bornes
supérieures sur le nombre d’états des automates produits, ce qui permet
d’estimer la complexité opérationnelle de ces constructions. En conclusion,
nous proposons des pistes pour des travaux futurs visant à affiner ces bornes
ou à déterminer leur optimalité.
