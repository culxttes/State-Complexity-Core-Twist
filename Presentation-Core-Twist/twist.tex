\section{Langage permuté}

\subsection{Définition}

\setframetitle{Définition}

\begin{frame}{\myframetitle}
  \begin{definition}[\(Twist(L)\)]
    Le langage \(Twist(L)\) sera défini comment l'ensemble des mots du
    langage de \(L\) tel qu'on aura échangé les lettres aux indices \(2k\) et
    \(2k + 1\) avec \(k\in \mathbb{N}\).
  \end{definition}

  \pause[]

  \begin{example}
    \vspace*{-\baselineskip}
    \begin{align*}
      L &= \{\varepsilon, a, ab, abcd\} \\
      Twist(L) &= \{\varepsilon, a, ba, badc\}
    \end{align*}
  \end{example}
\end{frame}

\subsection{Algorithme twister}

\setframetitle{Algorithme twister}

\begin{frame}{\myframetitle}
  \begin{example}[Soit l'automate \(M\) suivant]
    \centering
    \begin{tikzpicture}
      \tikzset{
        ->,
        >=stealth',
        node distance=2.5cm,
        every state/.style={thick},
        initial text=$ $,
      }
      \node[state, initial] (q1) {\(1\)};
      \node[state, right of=q1] (q2) {\(2\)};
      \node[state, accepting, right of=q2] (q3) {\(3\)};
      \node[state, above right of=q3] (q4) {\(4\)};
      \node[state, right of=q4] (q5) {\(5\)};
      \node[state, accepting, right of=q5] (q6) {\(6\)};
      \node[state, right of=q3] (q7) {\(7\)};
      \node[state, accepting, right of=q7] (q8) {\(8\)};
      \node[state, right of=q8] (q9) {\(9\)};
      \node[state, accepting, below right of=q3] (q10) {\(10\)};
      \node[state, right of=q10] (q11) {\(11\)};
      \node[state, right of=q11] (q12) {\(12\)};

      \only<1> {
        \draw
          (q1) edge[above] node[blue]{\(b\)} (q2)
          (q2) edge[above] node[red]{\(a\)} (q3)
          (q3) edge[above] node[blue]{\(b\)} (q4)
          (q4) edge[above] node[red]{\(a\)} (q5)
          (q5) edge[above] node[blue]{\(b\)} (q6)
          (q3) edge[above] node[blue]{\(a\)} (q7)
          (q7) edge[above] node[red]{\(b\)} (q8)
          (q8) edge[above] node[blue]{\(b\)} (q9)
          (q3) edge[above] node[blue]{\(b\)} (q10)
          (q10) edge[above] node[red]{\(b\)} (q11)
          (q11) edge[above] node[blue]{\(a\)} (q12);
      }
      \only<2> {
        \draw
          (q1) edge[above] node[red]{\(a\)} (q2)
          (q2) edge[above] node[blue]{\(b\)} (q3)
          (q3) edge[above] node[red]{\(a\)} (q4)
          (q4) edge[above] node[blue]{\(b\)} (q5)
          (q5) edge[above] node[blue]{\(b\)} (q6)
          (q3) edge[above] node[red]{\(b\)} (q7)
          (q7) edge[above] node[blue]{\(a\)} (q8)
          (q8) edge[above] node[blue]{\(b\)} (q9)
          (q3) edge[above] node[red]{\(b\)} (q10)
          (q10) edge[above] node[blue]{\(b\)} (q11)
          (q11) edge[above] node[blue]{\(a\)} (q12);
      }
    \end{tikzpicture}
  \end{example}
\end{frame}

\begin{frame}{\myframetitle}
  \begin{example}[Soit l'automate \(N\) suivant]
    \centering
    \only<1-2> {
      \begin{tikzpicture}
        \tikzset{
          ->,
          >=stealth',
          node distance=2.25cm,
          every state/.style={
            thick,
            inner sep = 0pt,
            minimum size = 0.75cm,
          },
          initial text=$ $,
        }
        \node[state, initial] (q1) {\(1\)};
        \node[state, right of=q1] (q2) {\(2\)};
        \node[state, above right of=q2] (q3) {\(3\)};
        \node[state, right of=q3] (q4) {\(4\)};
        \node[state, accepting, right of=q4] (q5) {\(5\)};
        \node[state, right of=q2] (q6) {\(6\)};
        \node[state, accepting, right of=q6] (q7) {\(7\)};
        \node[state, right of=q7] (q8) {\(8\)};

        \draw
          (q1) edge[above] node[blue]{\(b\)} (q2)
          (q2) edge[above] node[red]{\(a\)} (q3)
          (q3) edge[above] node[blue]{\(b\)} (q4)
          (q4) edge[above] node[red]{\(a\)} (q5)
          (q2) edge[above] node[red]{\(b\)} (q6)
          (q6) edge[above] node[blue]{\(a\)} (q7)
          (q7) edge[above] node[red]{\(b\)} (q8);
      \end{tikzpicture}

      \pause[]

      \vphantom{}

      \begin{tikzpicture}
        \tikzset{
          ->,
          >=stealth',
          node distance=2.25cm,
          every state/.style={
            thick,
            inner sep = 0pt,
            minimum size = 0.75cm,
          },
          initial text=$ $,
        }
        \node[state, initial] (q1) {\(1\)};
        \node[state, right of=q1] (q2) {\(2\)};
        \node[state, above right of=q2] (q3) {\(3\)};
        \node[state, right of=q3] (q4) {\(4\)};
        \node[state, accepting, right of=q4] (q5) {\(5\)};
        \node[state, right of=q2] (q6) {\(6\)};
        \node[state, accepting, right of=q6] (q7) {\(7\)};
        \node[state, right of=q7] (q8) {\(8\)};

        \draw
          (q1) edge[above] node[red]{\(a, b\)} (q2)
          (q2) edge[above] node[blue]{\(b\)} (q3)
          (q3) edge[above] node[red]{\(a\)} (q4)
          (q4) edge[above] node[blue]{\(b\)} (q5)
          (q2) edge[above] node[blue]{\(b\)} (q6)
          (q6) edge[above] node[red]{\(b\)} (q7)
          (q7) edge[above] node[blue]{\(a\)} (q8);
      \end{tikzpicture}
    }
    \only<3-4> {
      \begin{tikzpicture}
        \tikzset{
          ->,
          >=stealth',
          node distance=2.25cm,
          every state/.style={
            thick,
            inner sep = 0pt,
            minimum size = 0.75cm,
          },
          initial text=$ $,
        }
        \node[state, initial] (q1) {\(1\)};
        \node[state, right of=q1] (q2b) {\(2_b\)};
        \node[state, left of=q3] (q2a) {\(2_a\)};
        \node[state, above right of=q2b] (q3) {\(3\)};
        \node[state, right of=q3] (q4) {\(4\)};
        \node[state, accepting, right of=q4] (q5) {\(5\)};
        \node[state, right of=q2] (q6) {\(6\)};
        \node[state, accepting, right of=q6] (q7) {\(7\)};
        \node[state, right of=q7] (q8) {\(8\)};

        \draw
          (q1) edge[above] node[blue]{\(b\)} (q2a)
          (q1) edge[above] node[blue]{\(b\)} (q2b)
          (q2a) edge[above] node[red]{\(a\)} (q3)
          (q3) edge[above] node[blue]{\(b\)} (q4)
          (q4) edge[above] node[red]{\(a\)} (q5)
          (q2b) edge[above] node[red]{\(b\)} (q6)
          (q6) edge[above] node[blue]{\(a\)} (q7)
          (q7) edge[above] node[red]{\(b\)} (q8);
      \end{tikzpicture}

      \pause[4]

      \vphantom{}

      \begin{tikzpicture}
        \tikzset{
          ->,
          >=stealth',
          node distance=2.25cm,
          every state/.style={
            thick,
            inner sep = 0pt,
            minimum size = 0.75cm,
          },
          initial text=$ $,
        }
        \node[state, initial] (q1) {\(1\)};
        \node[state, right of=q1] (q2b) {\(2_b\)};
        \node[state, left of=q3] (q2a) {\(2_a\)};
        \node[state, above right of=q2b] (q3) {\(3\)};
        \node[state, right of=q3] (q4) {\(4\)};
        \node[state, accepting, right of=q4] (q5) {\(5\)};
        \node[state, right of=q2] (q6) {\(6\)};
        \node[state, accepting, right of=q6] (q7) {\(7\)};
        \node[state, right of=q7] (q8) {\(8\)};

        \draw
          (q1) edge[above] node[red]{\(a\)} (q2a)
          (q1) edge[above] node[red]{\(b\)} (q2b)
          (q2a) edge[above] node[blue]{\(b\)} (q3)
          (q3) edge[above] node[red]{\(a\)} (q4)
          (q4) edge[above] node[blue]{\(b\)} (q5)
          (q2b) edge[above] node[blue]{\(b\)} (q6)
          (q6) edge[above] node[red]{\(b\)} (q7)
          (q7) edge[above] node[blue]{\(a\)} (q8);
      \end{tikzpicture}
    }
  \end{example}
\end{frame}

\begin{frame}{\myframetitle}
  \begin{block}{Automate quelconque}
    Si on ajoute les cycles, on devra alors supposer que tous n\oe{}uds
    intermédiaire se trouve dans le cas du n\oe{}ud \(2\), on devra donc le
    dupliquer.
  \end{block}
\end{frame}

\begin{frame}{\myframetitle}
  \begin{example}
    \centering
    \only<1-2> {
      \begin{tikzpicture}
        \tikzset{
          ->,
          >=stealth',
          node distance=2.25cm,
          every state/.style={
            thick,
            inner sep = 0pt,
            minimum size = 0.75cm,
          },
          initial text=$ $,
        }
        \node[state, initial] (1) {\(1\)};
        \node[state, right of=1] (2) {\(2\)};
        \node[state, accepting, right of=2] (3) {\(3\)};
        \draw
          (1) edge[below] node{\(a\)} (2)
          (2) edge[below] node{\(b\)} (3)
          (3) edge[bend right, above] node{\(b\)} (1);
      \end{tikzpicture}
    }
    \only<2> {
      \begin{tikzpicture}
        \tikzset{
          ->,
          >=stealth',
          node distance=2.25cm,
          every state/.style={
            thick,
            inner sep = 0pt,
            minimum size = 0.75cm,
          },
          initial text=$ $,
        }
        \node[state, initial] (1) {\(1\)};
        \node[state, above of=3] (1a) {\(1_a\)};
        \node[state, right of=1] (2b) {\(2_b\)};
        \node[state, above of=2b] (2) {\(2\)};
        \node[state, accepting, right of=2b] (3) {\(3\)};
        \node[state, accepting, above of=1] (3b) {\(3_b\)};
        \draw
          (1) edge[below] node{\(b\)} (2b)
          (2b) edge[below] node{\(a\)} (3)
          (3) edge[right] node{\(a\)} (1a)
          (1a) edge[above] node{\(b\)} (2)
          (2) edge[above] node{\(b\)} (3b)
          (3b) edge[left] node{\(b\)} (1);
      \end{tikzpicture}
    }
  \end{example}
\end{frame}

\begin{frame}{\myframetitle}
  \begin{block}{Compléxité en état de notre algorithme}
    Dans le pire cas, notre algorithme dupliquera tous les états de
    l'automate. 

    \pause[]
    \vphantom{}

    Sachant que chaque état peut être dupliqué étant de voir qu'il y a de
    symbole de l'alphabet plus un. 

    \pause[]
    \vphantom{}

    Alors, la complexité dans le pire cas de notre algorithme est \(n(\lvert
    \Sigma \rvert + 1)\).
  \end{block}
\end{frame}
