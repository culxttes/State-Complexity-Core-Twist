% !chktex-file 8

% LTeX: enabled=false

\documentclass[14pt,aspectratio=169,table,french]{beamer}
\usepackage{bxdpx-beamer}
\usepackage[french]{babel}
\usepackage{caption}
\usepackage{hyperref}
\usepackage{bookmark}
\usepackage{luatexja}
\usepackage{luatexja-fontspec}
\usepackage{fontspec}
\usepackage[T1]{fontenc}
\usepackage{lmodern}
\usepackage{newtxtext}
\usepackage{amsmath}
\usepackage{breqn}
\usepackage{silence}
\usepackage{stmaryrd}
\usepackage{hyperref}
\usepackage{tikz}
\usepackage{xcolor}
\usepackage{minted}
\usepackage{xcolor}
\usepackage{csquotes}

\usetikzlibrary{arrows, arrows.meta}
\usetikzlibrary{automata}
\usetikzlibrary{positioning}
\usetikzlibrary{shapes.geometric, shapes.misc}
\usetikzlibrary{math}
\usetikzlibrary{calc}
\usetikzlibrary{fit}

\WarningFilter{latexfont}{Font shape}

\newcommand{\myframetitle}{}
\newcommand{\setframetitle}[1]{\renewcommand{\myframetitle}{#1}}

\tikzset{
  ->,
  >=stealth',
  node distance=3cm,
  every state/.style={
    thick,
    rounded rectangle
  },
  initial text=\( \),
}

% Light gray background for code blocks
\definecolor{bgcolor}{HTML}{F0F0F0}

% Gray for function in code blocks
\definecolor{mgray}{HTML}{666666}

% Minted configuration
\setminted{
  linenos,              % Enable line numbers
  numbersep=8pt,        % Space between line numbers and code
  xleftmargin=2em,      % Add margin so numbers stay inside the block
  bgcolor=bgcolor,      % Set the background color of the code block
  style=algol_nu,       % Disable syntax highlighting (use plain black text)
  frame=none,           % Add a frame around the code block
  framesep=5pt,         % Space between the frame and the code
  breaklines=true,      % Automatically break long lines
  escapeinside=||       % Escape to LaTeX between the two characters specified
}

\DeclareMathOperator*{\NFA}{NFA}
\DeclareMathOperator*{\DFA}{DFA}
\DeclareMathOperator*{\RAT}{Rat}
\DeclareMathOperator*{\ndet}{ndet}
\DeclareMathOperator*{\nsdet}{(n)det} % chktex 36

\setbeamertemplate{theorems}[numbered]

\usetheme{Boadilla}
\usefonttheme{professionalfonts}

\DeclareFontFamily{OMX}{mlmex}{}
\DeclareFontShape{OMX}{mlmex}{m}{n}{<->mlmex10}{}
\renewcommand{\familydefault}{\sfdefault}
\renewcommand{\kanjifamilydefault}{\gtdefault}

\usefonttheme{structurebold}

\setbeamerfont{alerted text}{series=\bfseries}
\setbeamerfont{section in toc}{series=\mdseries}
\setbeamerfont{frametitle}{size=\Large}
\setbeamerfont{title}{size=\LARGE}
\setbeamerfont{date}{size=\small}

\definecolor{UniBlue}{RGB}{0,150,200}
\definecolor{AlertOrange}{RGB}{255,76,0}
\definecolor{AlmostBlack}{RGB}{38,38,38}
\setbeamercolor{normal text}{fg=AlmostBlack}
\setbeamercolor{structure}{fg=UniBlue}
\setbeamercolor{block title}{fg=UniBlue!50!black}
\setbeamercolor{alerted text}{fg=AlertOrange}
\mode<beamer>{
  \definecolor{BackGroundGray}{RGB}{254,254,254}
  \setbeamercolor{background canvas}{bg=BackGroundGray}
}

\setbeamertemplate{blocks}[rounded]
\useinnertheme{circles}
\setbeamertemplate{navigation symbols}{}
\setbeamertemplate{footline}[frame number]
\setbeamertemplate{frametitle continuation}{}

\setbeamertemplate{title page}{
  \vspace{2.0em}
  {\usebeamerfont{title} \usebeamercolor[fg]{title} \inserttitle~\par}
  {\usebeamerfont{subtitle}\usebeamercolor[fg]{subtitle}\insertsubtitle~\par}
  \vspace{2.5em}
  \begin{minipage}{0.48\textwidth}
    \usebeamercolor[fg]{titlegraphic}\inserttitlegraphic{}
  \end{minipage}
  \hfill
  \begin{minipage}{0.48\textwidth}
    \begin{flushright}
      \usebeamerfont{author}\insertauthor\par
      \usebeamerfont{institute}\insertinstitute~\par
      \vspace{3em}
      \usebeamerfont{date}\insertdate\par
    \end{flushright}
  \end{minipage}
}

\newcounter{firstsection}
\setcounter{firstsection}{1}

\AtBeginSection[]{
  \ifnum\value{firstsection}=1
  \setcounter{firstsection}{0}
  \begin{frame}
    \tableofcontents[hideallsubsections]
  \end{frame}
  \else
  \begin{frame}
    \centering
    \usebeamerfont{title}
    \usebeamercolor[fg]{title}
    \insertsection{}
  \end{frame}
  \fi
}

% LTeX: enabled=true

\title{Complexité en état opérationnel de langage rationnel}
\subtitle{
  Master 1 - ITA~: Application Informatique \\ Encadré par Pascal Caron
}
\author{Edouard HADDAG}
\institute{Université de Rouen Normandie}
\date{\today}
\titlegraphic{
  \includegraphics[scale=0.3]{
    resources/logo-univ-rouen-normandie-noir.png
  }
}

\begin{document}

\maketitle

% !chktex-file 8

\section{Introduction}

\subsection{Prémisse}

\subsubsection{Langage}

\setframetitle{Langage}

\begin{frame}{\myframetitle}
  \begin{definition}[Langage]
    Un << langage >> \(L\) est un ensemble de mots sur un alphabet \(\Sigma\).
  \end{definition}

  \pause[]

  \begin{example}
    \vspace{-1.5\topsep}
    \begin{gather*}
      L = \{\varepsilon, aa, bb, abab\}
    \end{gather*}
  \end{example}

  \pause[]

  \begin{definition}[Le langage de l'ensemble des mots]
    L'ensemble des mots possible sur l'alphabet \(\Sigma\) sera noté
    \(\Sigma^*\).
  \end{definition}
\end{frame}

\subsubsection{Automate}

\setframetitle{Automate}

\begin{frame}{\myframetitle}
  \begin{definition}[Automate]
    \(M = (Q, I, F, \delta)\) avec \(Q = \{q_1, q_2, q_3\}\), \(I = \{q_1\}\)
    et \(F = \{q_3\}\).
    \begin{center}
      \begin{tikzpicture}
        \node[state, initial] (q1) {\(q_1\)};
        \node[state, right of=q1] (q2) {\(q_2\)};
        \node[state, accepting, right of=q2] (q3) {\(q_3\)};

        \draw
        (q1) edge[loop above] node{\(a\)} (q1)
        (q1) edge[above] node{\(b\)} (q2)
        (q1) edge[bend right, below] node{\(a\)} (q3)
        (q2) edge[above] node{\(a,b\)} (q3)
        (q3) edge[bend right=50, above] node{\(b\)} (q2);
      \end{tikzpicture}
    \end{center}
  \end{definition}
\end{frame}

\begin{frame}{\myframetitle}
  \begin{definition}[L'acceptation d'un mot]
    Le mot \(aba\) est << accepté >>~:
    \begin{center}
      \begin{tikzpicture}
        \node[state, initial] (q1) {\(q_1\)};
        \node[state, right of=q1] (q2) {\(q_2\)};
        \node[state, accepting, right of=q2] (q3) {\(q_3\)};

        \only<1> {\draw (q1) edge[loop above] node{\(a\)} (q1);}
        \only<1-2> {\draw (q1) edge[above] node{\(b\)} (q2);}
        \only<1-3> {\draw (q2) edge[above] node{\(a,b\)} (q3);}

        \only<2-> {\draw (q1) edge[loop above, red] node{\(a\)} (q1);}
        \only<3-> {\draw (q1) edge[above, red] node{\(b\)} (q2);}
        \only<4-> {\draw (q2) edge[above, red] node{\(a,b\)} (q3);}

        \draw
        (q3) edge[bend right=50, above] node{\(b\)} (q2)
        (q1) edge[bend right, below] node{\(a\)} (q3);
      \end{tikzpicture}
    \end{center}
  \end{definition}
\end{frame}

\begin{frame}{\myframetitle}
  \begin{definition}[Automate minimal]
    L'automate \(M_1\) est << minimal >> du langage \(L = \{ab\}\)~:

    \begin{minipage}{0.47\textwidth}
      \centering
      \begin{tikzpicture}[node distance=2.5cm]
        \node[state, initial] (0) {\(0\)};
        \node[state, right of=0] (1) {\(1\)};
        \node[state, accepting, below of=1] (2) {\(2\)};

        \node[above =.75cm of $(0)!0.5!(1)$, align=center] {\(M_1\)};

        \draw
        (0) edge[above] node{\(a\)} (1)
        (1) edge[right] node{\(b\)} (2);
      \end{tikzpicture}
    \end{minipage}
    \hfill
    \begin{minipage}{0.47\textwidth}
      \centering
      \begin{tikzpicture}[node distance=2.5cm]
        \node[state, initial] (0) {\(0\)};
        \node[state, right of=0] (1) {\(1\)};
        \node[state, accepting, below of=1] (2) {\(2\)};
        \node[state, left of=2] (3) {\(3\)};

        \node[above =.75cm of $(0)!0.5!(1)$, align=center] {\(M_2\)};

        \draw
        (0) edge[above] node{\(a\)} (1)
        (1) edge[right] node{\(b\)} (2)
        (0) edge[left] node{\(a\)} (3);
      \end{tikzpicture}
    \end{minipage}
  \end{definition}
\end{frame}

\subsubsection{Lien entre les langages et les automates}

\setframetitle{Lien entre les langages et les automates}

\begin{frame}{\myframetitle}
  \begin{definition}[Langage rationnel]
    Les langages << rationnels >> sont les langages reconnaissables par au
    moins un automate.
  \end{definition}
\end{frame}

\subsubsection{Complexité en état}

\setframetitle{Complexité en état}

\begin{frame}{\myframetitle}
  \begin{definition}
    La << complexité en état >> est une mesure d'un \textbf{langage}, elle est
    définie comme le nombre d'états de l'automate minimal du langage. Qui sera
    noté \(\mathcal{C}(L)\).
  \end{definition}
\end{frame}

\begin{frame}{\myframetitle}
  \begin{example}[%
      \only<1>{Définition d'une famille}%
      \only<2>{\(\mathcal{C}(A_2) = 2 + 2\)}%
    ]
    \centering
    \vspace{-1.5\topsep}
    \begin{gather*}
      A_n = \Sigma^* \cdot a \cdot \Sigma^n
    \end{gather*}
    \vspace{-2.25\topsep}

    \onslide<2-> {
      \begin{tikzpicture}[node distance=2.5cm, scale=0.7,transform shape]
        \node[state, initial] (0') {\(0\)};
        \node[state, right of=0'] (1') {\(1\)};
        \node[state, right of=1'] (2') {\(2\)};
        \node[state, accepting, right of=2'] (3') {\(3\)};

        \draw
        (0') edge[above] node{\(a\)} (1')
        (1') edge[above] node{\(a,b\)} (2')
        (2') edge[above] node{\(a,b\)} (3')
        (0') edge[loop above] node{\(a,b\)} (0');
      \end{tikzpicture}
    }
  \end{example}
\end{frame}

\begin{frame}{\myframetitle}
  \begin{definition}[Complexité en état opérationniel]
    La fonction \(f\) à une complexité \(g(n)\) si~:
    \begin{center}
      \begin{tikzpicture}[transform shape, scale=0.8]
        \node[draw, ellipse, minimum width=1.5cm] (A) at (-3,0) {\(A\)};
        \node[draw, rectangle, minimum width=2cm, minimum height=1cm]
        (F) at (0,0) {\(f\)};
        \node[draw, ellipse, minimum width=1.5cm] (C) at (3,0) {\(C\)};

        \node[left =0.25cm of A] {\(n\)};
        \node[right =0.25cm of C] {\(k \leq g(n)\)};

        \draw (A) -- (F);
        \draw (F) -- (C);
      \end{tikzpicture}
    \end{center}
  \end{definition}
\end{frame}

\section{Le trognon d'un langage}

Dans cette partie, nous nous concentrerons sur un langage bien particulier, le
<< trognon >>. Dans un premier temps, nous définirons formellement ce langage.
Enfin, nous présenterons un algorithme de << grignotage >> permettant de le
calculer à partir d'un automate non déterministe, l'automate du trognon du
langage de l'automate initial. Nous présenterons ensuite une preuve de cet
algorithme ainsi que la complexité dans le pire des cas de celui-ci. Enfin,
pour finir, nous conclurons cette partie en faisant une courte synthèse de ce
qu'on aura vu et de la possible continuation.

\subsection{Définition}

\begin{definition}[Le langage trognon]
  Le langage \(Core(L)\) (aussi noté \(Trognon(L)\)) est construit à partir de
  \(L\) en retirant, pour chaque mot de \(L\), un préfixe dont le miroir est
  également un suffixe~; ce suffixe est, lui aussi, retiré~:
  \begin{align*}
    Core(w) &= \{v \in \Sigma^* \mid \exists u \in \Sigma^* \land w = u \cdot v
    \cdot \overleftarrow{u}\}, \\
    Core(L) &= \bigcup_{w \in L} Core(w).
  \end{align*}
\end{definition}

\begin{example}
  \vspace{-\baselineskip}
  \begin{align*}
    L &= \{a, b, aa, aabcaa\}, \\
    Core(L) &= \{\varepsilon, a, b, bc, abca, aabcaa\}.
  \end{align*}
\end{example}

Maintenant que nous savons ce qu’est le trognon d’un langage, nous allons
présenter un algorithme permettant de construire un automate le reconnaissant,
à partir d’un automate reconnaissant le langage initial.

\subsection{Algorithme de grignotage}

Notre algorithme prendra en entrée un automate non déterministe et renverra un
automate non déterministe. Mais, avant de rentrer dans le dur de l'algorithme,
nous regardons le principe de l'algorithme, puis nous verrons sa définition
formelle.

\subsubsection{Principe de l'algorithme}

Le principe de cet algorithme est assez simple, si on prend l'automate 
suivant~:
\begin{figure}[H]
  \centering
  \captionsetup{type=figure,justification=centering}
  \begin{tikzpicture}
    \tikzset{
      ->,
      >=stealth',
      node distance=2.5cm,
      every state/.style={thick},
      initial text=$ $,
    }
    \node[state, initial] (q1) {\(q_1\)};
    \node[state, initial right, accepting, right of=q1] (q2) {\(q_2\)};
    \node[state, above of=q2] (q4) {\(q_4\)};
    \node[state, accepting, right of=q2] (q3) {\(q_3\)};
    \node[state, right of=q4] (q5) {\(q_5\)};

    \draw
      (q1) edge[above] node{a} (q4)
      (q1) edge[below] node{a} (q2)
      (q4) edge[bend right, below] node{b} (q3)
      (q4) edge[above] node{a} (q5)
      (q3) edge[above] node{b} (q4)
      (q3) edge[loop right] node{a} (q3)
      (q5) edge[bend right, above] node{b} (q4)
      (q5) edge[loop right] node{b} (q5);
  \end{tikzpicture}
  \caption{Exemple d'un automate avant grignotage.}
\end{figure}

Nous devons alors le << grignoté >> de tous les mots pour en faire l'automate
reconnaissant le trognon du langage. Avant d'essayer de le grignoter de tous
les mots, on va tenter de le grignoter d'un seul mot, \(ab\). On va donc
simplement << avancer >> les états initiaux de la transition \(ab\) et 
<< reculer >> les états finaux de la transition \(ab\)~:
\begin{align*}
  I' &= \bigcup_{i \in I} \delta(i, ab) = \{q_3\}, \\
  F' &= \bigcup_{f \in F} \delta^{-1}(f, ab) = \{q_4\}.
\end{align*}
avec \(\forall p \in \delta^{-1}(q, w) \Longrightarrow q \in
\delta(p, w)\).

\vphantom{}

Et, le reste de l'automate ne changerait pas, ce qui nous donnerait à la fin~:
\begin{figure}[H]
  \centering
  \captionsetup{type=figure,justification=centering}
  \begin{tikzpicture}
    \tikzset{
      ->,
      >=stealth',
      node distance=2.5cm,
      every state/.style={thick},
      initial text=$ $,
    }
    \node[state] (q1) {\(q_1\)};
    \node[state, right of=q1] (q2) {\(q_2\)};
    \node[state, accepting, above of=q2] (q4) {\(q_4\)};
    \node[state, initial above, right of=q2] (q3) {\(q_3\)};
    \node[state, right of=q4] (q5) {\(q_5\)};

    \draw
      (q1) edge[above] node{a} (q4)
      (q1) edge[below] node{a} (q2)
      (q4) edge[bend right, below] node{b} (q3)
      (q4) edge[above] node{a} (q5)
      (q3) edge[above] node{b} (q4)
      (q3) edge[loop right] node{a} (q3)
      (q5) edge[bend right, above] node{b} (q4)
      (q5) edge[loop right] node{b} (q5);
  \end{tikzpicture}
  \caption{Exemple d'un automate après grignotage de \(ab\).}
\end{figure}

\begin{definition}[Grignotage d'un automate par un mot]
  Soit \(M = (Q, I, F, \delta)\), un automate, le \emph{grignotage} de \(M\)
  par un mot \(w\) sera l'automate~:
  \begin{gather*}
    Nibbling(M, w) = (Q, I', F', \delta), \\
    I' = \bigcup_{i \in I} \delta(i, w) \land F' = \bigcup_{f \in F}
    \delta^{-1}(f, w). \\
  \end{gather*}
\end{definition}

\begin{lemma}[Premier lemme du grignotage]\label{lemma:grignotage}
  Si un mot \(w\) est reconnu par \(Nibbling(M, v)\), alors \(v \cdot w \cdot
  \overleftarrow{w}\) est reconnu par M.
  \begin{align*}
    \forall M \in NFA(\Sigma, \eta) \land \forall v \in \Sigma^* 
    \land N = Nibbling(M, v) \land w \in L(N) \Longleftrightarrow v \cdot w
    \cdot \overleftarrow{v} \in L(M).
  \end{align*}
\end{lemma}

\begin{lemma}[Second lemme du grignotage]\label{lemma:grignotage2}
  Soit un automate \(M \in NFA(\Sigma, \eta)\), et un mot \(w = u \cdot v
  \cdot \overleftarrow{u} \in L(M)\), si~:
  \begin{align*}
  \exists (x, y, z) \in (\Sigma^*)^3 \colon Nibbling(M, y) = Nibbling(M, x)
  \land u = x \cdot z \Longrightarrow (y \cdot z) \cdot v \cdot
  \overleftarrow{(y \cdot z)} \in L(M).
  \end{align*}
\end{lemma}

Il nous reste donc à présent à calculer le grignoté de tous les mots dits
<< pertinents >>, puis à effectuer l'union non déterministe des automates
ainsi obtenus.

\vphantom{}

Cependant, une question importante demeure~: quels sont les mots que l’on
qualifie de pertinents~?

\vphantom{}

Une première intuition serait de considérer tous les fragments internes des
mots du langage, autrement dit toutes les sous-parties situées entre le début
et la fin, ce que l’on pourrait appeler, de manière imagée, \textit{la chair}
du mot (par opposition au trognon).

\vphantom{}

Mais, cette approche soulève un problème~: le langage étant, en général,
infini, il existe potentiellement une infinité de << chairs >>. Une telle
démarche, si elle n’est pas restreinte, est alors incalculable en pratique.

\begin{definition}[La chair d'un langage]
  Le langage \(Flesh(L)\) (aussi noté \(Chair(L)\)) est construit à partir de
  \(L\) en gardant, pour chaque mot de \(L\), un préfixe dont le miroir est
  également un suffixe~:
  \begin{align*}
    Flesh(w) &= \{u \in \Sigma^* \mid \exists v \in \Sigma^* \land w = u
    \cdot v \cdot \overleftarrow{u}\}, \\
    Flesh(L) &= \bigcup_{w \in L} Flesh(w).
  \end{align*}
\end{definition}

\begin{example}
  \vspace{-\baselineskip}
  \begin{align*}
    L &= \{a, b, aa, aabcaa\}, \\
    Flesh(L) &= \{\varepsilon, a, b, aa\}.
  \end{align*}
\end{example}

Puisque l’automate reconnaissant le langage de départ est fini, notons \(n\)
son nombre d’états. Si un mot \(u\) appartient à la chair d’un mot, et que sa
longueur vérifie \(\lvert u \rvert \geq n\), alors la lecture de \(u\) force
la réapparition d’au moins un état (par le \textit{principe des tiroirs}). Il
en résulte l’existence d’un cycle dans l'automate de départ.

\vphantom{}

L’algorithme explore donc tous les mots \(w \in \Sigma^*\) en construisant
successivement les automates grignotés \(N = Nibbling(M, w)\). Puisque le
nombre d’automates à états fixes est fini, on aboutit forcément à une
répétition~: il existe un mot plus court \(u\) tel que
\begin{align*}
N = Nibbling(M, u) \land \lvert u \rvert < \lvert w \rvert.
\end{align*}

À partir de ce moment, il n’est plus nécessaire << d’étendre >> \(w\), car
tout mot préfixé par \(w\) aurait déjà été obtenu via un préfixe plus court.
Cette condition garantit l’arrêt de l’algorithme et garantit que l’on ne
construit qu’un nombre fini d’automates.

\subsubsection{Formalisation de l'algorithme}

Maintenant que nous avons l'idée de l'algorithme, nous verrons sa
définition formelle~:
\begin{minted}{text}
nibbling(M):
  N |\leftarrow| M
  la |\leftarrow \{\,\texttt{M}\,\}|
  lw |\leftarrow| queue_empty()
  enqueue(lw, |\varepsilon|)
  |\textbf{while}| lw |\!\(\neq \varnothing\)| |\textbf{do}|
    |\textbf{for}| s |\textbf{in}| |\Sigma| |\textbf{do}|
      w |\leftarrow| queue_dequeue(lw) |\!\cdot\!| s
      A |\leftarrow \(Nibbling(\,\texttt{M},\; \texttt{w}\,)\)|
      |\textbf{if}| A.I |\(\!\neq \varnothing\) \land| A.F |\(\!\neq \varnothing\) \land| A |\!\(\notin\)\!| la |\textbf{then}|
        N |\leftarrow| N |\cup| A
        la |\leftarrow \texttt{la} \cup \{\,\texttt{A}\,\}|
        enqueue(lw, w)
  |\textbf{return}| N
\end{minted}

\begin{remark}
  On observe que l’algorithme effectue un parcours en largeur de
  l’arborescence des mots. Cela garantit que tous les mots de longueur
  strictement inférieure ont été explorés avant de commencer à << grignoter >>
  ceux de longueur supérieure.
\end{remark}

\subsection{Preuve de notre algorithme}

Pour prouver que notre algorithme est correct et construit bien l'automate
du trognon d'un langage, nous allons d'abord montrer que tous mots reconnus
par l'automate sont des mots appartenant au trognon du langage. Enfin, nous
allons montrer que tous les mots du trognon d'un langage sont reconnu par
l'automate.

\subsubsection{Tous les mots reconnus sont des mots du trognon}

\begin{proof}
  Soit \(M \in NFA(\Sigma, \eta)\), \(N = \texttt{nibbling}(M)\) et un mot
  \(w\) reconnu par l'automate \(N\). S'il est reconnu par \(N\), c'est qu'il
  est reconnu par au moins un automate \(Nibbling(M, v)\) qui compose \(N\).
  Or, par le lemme~\ref{lemma:grignotage}, alors \(v \cdot w \cdot
  \overleftarrow{v}\) est reconnu par \(M\), et donc \(w\) fait partie de
  \(Core(L(M))\).
\end{proof}

\subsubsection{Tous les mots du trognon sont reconnus}

\begin{proof}
  Pour prouver que tous les mots du trognon de \(M \in NFA(\Sigma, \eta)\)
  sont bien reconnus par l'automate \(N = \texttt{nibbling}(M)\), nous allons
  faire une preuve par récurrence sur la longueur de la chair d'un mot \(w =
  u \cdot v \cdot \overleftarrow{u}\) du langage \(L(M)\)~:

  \vphantom{}

  \begin{itemize}
    \item[\bullet~\textbf{Initialisation~:}] \(\lvert u \rvert = 0\)

      Si \(\lvert u \rvert = 0 \Longleftrightarrow u = \varepsilon
      \Longleftrightarrow v = w\).

      \vphantom{}
      
      Ensuite, puisque nous faisons l'union de l'automate \(M\) dans notre
      algorithme~:
      \begin{align*}
        w \in L(M) \Longleftrightarrow v \in L(M) \Longleftrightarrow v \in
        L(\texttt{nibbling}(M)).
      \end{align*}

    \item[\bullet~\textbf{Hérédité~:}] \(\lvert u \rvert \geq 1\)
      \begin{itemize}
        \item[\circ] Si \(Nibbling(M, u)\) fait partie des automates dont on a
          fait l'union dans notre algorithme, alors par le
          lemme~\ref{lemma:grignotage}~:
          \begin{align*}
              v \in L(Nibbling(M, u)) \Longleftrightarrow v \in
              L(\texttt{nibbling}(M)).
          \end{align*}

        \item[\circ] Sinon cela veut dire que~:
          \begin{align*}
            \exists (y, x, z) \in \Sigma^* \colon Nibbling(M, y) = 
            Nibbling(M, x) \land u = x \cdot z \land \lvert y \rvert < \lvert x 
            \rvert.
          \end{align*}
          On s'intéresse donc au mot \(w' =  (y \cdot z) \cdot v \cdot
          \overleftarrow{(y \cdot z)}\) qui est dans \(L(M)\) par le
          lemme~\ref{lemma:grignotage2}, or par hypothèse de récurrence
          (\(\lvert y \cdot z \rvert < \lvert u \rvert\)) \(v \in
          L(\texttt{nibbling}(M))\).
      \end{itemize}
  \end{itemize}
\end{proof}

\newpage
\subsection{Pire cas de notre algorithme}

Dans cette section, nous établirons une borne supérieure sur le nombre
d’états de l’automate que notre algorithme peut engendrer. Cette estimation,
distincte de la complexité du trognon d’un langage, nous offre néanmoins un
plafond pour la complexité en états.

\vphantom{}

Dans notre pire cas, notre algorithme fera l'union de tous les automates
distincts qu'ont les mêmes états et la même fonction de transition. La seule
chose qui peut changer c'est l'ensemble des états initiaux et l'ensemble des
états finaux.

\vphantom{}

On sait que ces deux ensembles sont des ensembles finis non vide à maximum
\(n\) élément, il y a donc \(2^n - 1\) ensembles d'états initiaux et \(2^n -
1\) ensembles d'états finaux. Ce qui nous donne \((2^n - 1)^2\) automate
différents. Ensuite, puisqu'on fait l'union non déterministe de deux
automates, l'automate résultant de notre algorithme aurait dans le pire des
cas \(n(2^n - 1)^2\) états avec \(n\) le nombre d'états de l'automate de
départ.

\subsection{Conclusion}

Pour conclure, nous venons de créer un algorithme qui prendre n'importe quel
automate en entrée et qui calcule l'automate du trognon du langage de cette
automate. De plus, notre algorithme, créer un automate dans le pire des cas à
\(n(2^n - 1)^2\) états. Par conséquent, la complexité en états du grignotage
d'un langage est d'au moins \(n(2^n - 1)^2\).

\vphantom{}

En exécutant cet algorithme, nous pouvons remarquer que les automates produit
ne sont pas minimaux, ce qui nous laisse penser que cette complexité en états
n'est pas atteinte. Néanmoins, par manque de temps, aucune preuve de cette
hypothèse sera fourni.

\vphantom{}

Il serait donc intéressant d'étudier la complexité en état du grignotage plus
en détail dans un autre article.

\section{Langage permuté}

\subsection{Définition}

\setframetitle{Définition}

\begin{frame}{\myframetitle}
  \begin{definition}[\(Twist(L)\)]
    Le langage \(Twist(L)\) sera défini comment l'ensemble des mots du
    langage de \(L\) tel qu'on aura échangé les lettres aux indices \(2k\) et
    \(2k + 1\) avec \(k\in \mathbb{N}\).
  \end{definition}

  \pause[]

  \begin{example}
    \vspace*{-\baselineskip}
    \begin{align*}
      L &= \{\varepsilon, a, ab, abcd\} \\
      Twist(L) &= \{\varepsilon, a, ba, badc\}
    \end{align*}
  \end{example}
\end{frame}

\subsection{Algorithme twister}

\setframetitle{Algorithme twister}

\begin{frame}{\myframetitle}
  \begin{example}[Soit l'automate \(M\) suivant]
    \centering
    \begin{tikzpicture}
      \tikzset{
        ->,
        >=stealth',
        node distance=2.5cm,
        every state/.style={thick},
        initial text=$ $,
      }
      \node[state, initial] (q1) {\(1\)};
      \node[state, right of=q1] (q2) {\(2\)};
      \node[state, accepting, right of=q2] (q3) {\(3\)};
      \node[state, above right of=q3] (q4) {\(4\)};
      \node[state, right of=q4] (q5) {\(5\)};
      \node[state, accepting, right of=q5] (q6) {\(6\)};
      \node[state, right of=q3] (q7) {\(7\)};
      \node[state, accepting, right of=q7] (q8) {\(8\)};
      \node[state, right of=q8] (q9) {\(9\)};
      \node[state, accepting, below right of=q3] (q10) {\(10\)};
      \node[state, right of=q10] (q11) {\(11\)};
      \node[state, right of=q11] (q12) {\(12\)};

      \only<1> {
        \draw
          (q1) edge[above] node[blue]{\(b\)} (q2)
          (q2) edge[above] node[red]{\(a\)} (q3)
          (q3) edge[above] node[blue]{\(b\)} (q4)
          (q4) edge[above] node[red]{\(a\)} (q5)
          (q5) edge[above] node[blue]{\(b\)} (q6)
          (q3) edge[above] node[blue]{\(a\)} (q7)
          (q7) edge[above] node[red]{\(b\)} (q8)
          (q8) edge[above] node[blue]{\(b\)} (q9)
          (q3) edge[above] node[blue]{\(b\)} (q10)
          (q10) edge[above] node[red]{\(b\)} (q11)
          (q11) edge[above] node[blue]{\(a\)} (q12);
      }
      \only<2> {
        \draw
          (q1) edge[above] node[red]{\(a\)} (q2)
          (q2) edge[above] node[blue]{\(b\)} (q3)
          (q3) edge[above] node[red]{\(a\)} (q4)
          (q4) edge[above] node[blue]{\(b\)} (q5)
          (q5) edge[above] node[blue]{\(b\)} (q6)
          (q3) edge[above] node[red]{\(b\)} (q7)
          (q7) edge[above] node[blue]{\(a\)} (q8)
          (q8) edge[above] node[blue]{\(b\)} (q9)
          (q3) edge[above] node[red]{\(b\)} (q10)
          (q10) edge[above] node[blue]{\(b\)} (q11)
          (q11) edge[above] node[blue]{\(a\)} (q12);
      }
    \end{tikzpicture}
  \end{example}
\end{frame}

\begin{frame}{\myframetitle}
  \begin{example}[Soit l'automate \(N\) suivant]
    \centering
    \only<1-2> {
      \begin{tikzpicture}
        \tikzset{
          ->,
          >=stealth',
          node distance=2.25cm,
          every state/.style={
            thick,
            inner sep = 0pt,
            minimum size = 0.75cm,
          },
          initial text=$ $,
        }
        \node[state, initial] (q1) {\(1\)};
        \node[state, right of=q1] (q2) {\(2\)};
        \node[state, above right of=q2] (q3) {\(3\)};
        \node[state, right of=q3] (q4) {\(4\)};
        \node[state, accepting, right of=q4] (q5) {\(5\)};
        \node[state, right of=q2] (q6) {\(6\)};
        \node[state, accepting, right of=q6] (q7) {\(7\)};
        \node[state, right of=q7] (q8) {\(8\)};

        \draw
          (q1) edge[above] node[blue]{\(b\)} (q2)
          (q2) edge[above] node[red]{\(a\)} (q3)
          (q3) edge[above] node[blue]{\(b\)} (q4)
          (q4) edge[above] node[red]{\(a\)} (q5)
          (q2) edge[above] node[red]{\(b\)} (q6)
          (q6) edge[above] node[blue]{\(a\)} (q7)
          (q7) edge[above] node[red]{\(b\)} (q8);
      \end{tikzpicture}

      \pause[]

      \vphantom{}

      \begin{tikzpicture}
        \tikzset{
          ->,
          >=stealth',
          node distance=2.25cm,
          every state/.style={
            thick,
            inner sep = 0pt,
            minimum size = 0.75cm,
          },
          initial text=$ $,
        }
        \node[state, initial] (q1) {\(1\)};
        \node[state, right of=q1] (q2) {\(2\)};
        \node[state, above right of=q2] (q3) {\(3\)};
        \node[state, right of=q3] (q4) {\(4\)};
        \node[state, accepting, right of=q4] (q5) {\(5\)};
        \node[state, right of=q2] (q6) {\(6\)};
        \node[state, accepting, right of=q6] (q7) {\(7\)};
        \node[state, right of=q7] (q8) {\(8\)};

        \draw
          (q1) edge[above] node[red]{\(a, b\)} (q2)
          (q2) edge[above] node[blue]{\(b\)} (q3)
          (q3) edge[above] node[red]{\(a\)} (q4)
          (q4) edge[above] node[blue]{\(b\)} (q5)
          (q2) edge[above] node[blue]{\(b\)} (q6)
          (q6) edge[above] node[red]{\(b\)} (q7)
          (q7) edge[above] node[blue]{\(a\)} (q8);
      \end{tikzpicture}
    }
    \only<3-4> {
      \begin{tikzpicture}
        \tikzset{
          ->,
          >=stealth',
          node distance=2.25cm,
          every state/.style={
            thick,
            inner sep = 0pt,
            minimum size = 0.75cm,
          },
          initial text=$ $,
        }
        \node[state, initial] (q1) {\(1\)};
        \node[state, right of=q1] (q2b) {\(2_b\)};
        \node[state, left of=q3] (q2a) {\(2_a\)};
        \node[state, above right of=q2b] (q3) {\(3\)};
        \node[state, right of=q3] (q4) {\(4\)};
        \node[state, accepting, right of=q4] (q5) {\(5\)};
        \node[state, right of=q2] (q6) {\(6\)};
        \node[state, accepting, right of=q6] (q7) {\(7\)};
        \node[state, right of=q7] (q8) {\(8\)};

        \draw
          (q1) edge[above] node[blue]{\(b\)} (q2a)
          (q1) edge[above] node[blue]{\(b\)} (q2b)
          (q2a) edge[above] node[red]{\(a\)} (q3)
          (q3) edge[above] node[blue]{\(b\)} (q4)
          (q4) edge[above] node[red]{\(a\)} (q5)
          (q2b) edge[above] node[red]{\(b\)} (q6)
          (q6) edge[above] node[blue]{\(a\)} (q7)
          (q7) edge[above] node[red]{\(b\)} (q8);
      \end{tikzpicture}

      \pause[4]

      \vphantom{}

      \begin{tikzpicture}
        \tikzset{
          ->,
          >=stealth',
          node distance=2.25cm,
          every state/.style={
            thick,
            inner sep = 0pt,
            minimum size = 0.75cm,
          },
          initial text=$ $,
        }
        \node[state, initial] (q1) {\(1\)};
        \node[state, right of=q1] (q2b) {\(2_b\)};
        \node[state, left of=q3] (q2a) {\(2_a\)};
        \node[state, above right of=q2b] (q3) {\(3\)};
        \node[state, right of=q3] (q4) {\(4\)};
        \node[state, accepting, right of=q4] (q5) {\(5\)};
        \node[state, right of=q2] (q6) {\(6\)};
        \node[state, accepting, right of=q6] (q7) {\(7\)};
        \node[state, right of=q7] (q8) {\(8\)};

        \draw
          (q1) edge[above] node[red]{\(a\)} (q2a)
          (q1) edge[above] node[red]{\(b\)} (q2b)
          (q2a) edge[above] node[blue]{\(b\)} (q3)
          (q3) edge[above] node[red]{\(a\)} (q4)
          (q4) edge[above] node[blue]{\(b\)} (q5)
          (q2b) edge[above] node[blue]{\(b\)} (q6)
          (q6) edge[above] node[red]{\(b\)} (q7)
          (q7) edge[above] node[blue]{\(a\)} (q8);
      \end{tikzpicture}
    }
  \end{example}
\end{frame}

\begin{frame}{\myframetitle}
  \begin{block}{Automate quelconque}
    Si on ajoute les cycles, on devra alors supposer que tous n\oe{}uds
    intermédiaire se trouve dans le cas du n\oe{}ud \(2\), on devra donc le
    dupliquer.
  \end{block}
\end{frame}

\begin{frame}{\myframetitle}
  \begin{example}
    \centering
    \only<1-2> {
      \begin{tikzpicture}
        \tikzset{
          ->,
          >=stealth',
          node distance=2.25cm,
          every state/.style={
            thick,
            inner sep = 0pt,
            minimum size = 0.75cm,
          },
          initial text=$ $,
        }
        \node[state, initial] (1) {\(1\)};
        \node[state, right of=1] (2) {\(2\)};
        \node[state, accepting, right of=2] (3) {\(3\)};
        \draw
          (1) edge[below] node{\(a\)} (2)
          (2) edge[below] node{\(b\)} (3)
          (3) edge[bend right, above] node{\(b\)} (1);
      \end{tikzpicture}
    }
    \only<2> {
      \begin{tikzpicture}
        \tikzset{
          ->,
          >=stealth',
          node distance=2.25cm,
          every state/.style={
            thick,
            inner sep = 0pt,
            minimum size = 0.75cm,
          },
          initial text=$ $,
        }
        \node[state, initial] (1) {\(1\)};
        \node[state, above of=3] (1a) {\(1_a\)};
        \node[state, right of=1] (2b) {\(2_b\)};
        \node[state, above of=2b] (2) {\(2\)};
        \node[state, accepting, right of=2b] (3) {\(3\)};
        \node[state, accepting, above of=1] (3b) {\(3_b\)};
        \draw
          (1) edge[below] node{\(b\)} (2b)
          (2b) edge[below] node{\(a\)} (3)
          (3) edge[right] node{\(a\)} (1a)
          (1a) edge[above] node{\(b\)} (2)
          (2) edge[above] node{\(b\)} (3b)
          (3b) edge[left] node{\(b\)} (1);
      \end{tikzpicture}
    }
  \end{example}
\end{frame}

\begin{frame}{\myframetitle}
  \begin{block}{Compléxité en état de notre algorithme}
    Dans le pire cas, notre algorithme dupliquera tous les états de
    l'automate. 

    \pause[]
    \vphantom{}

    Sachant que chaque état peut être dupliqué étant de voir qu'il y a de
    symbole de l'alphabet plus un. 

    \pause[]
    \vphantom{}

    Alors, la complexité dans le pire cas de notre algorithme est \(n(\lvert
    \Sigma \rvert + 1)\).
  \end{block}
\end{frame}

\section{Conclusion}

Durant cet article, nous avons défini deux transformations particulières
de langages réguliers~: le trognon d’un langage et le langage permuté. Pour
chacune d’elles, nous avons conçu un algorithme capable de construire, à
partir d’un automate non déterministe initial, un nouvel automate
reconnaissant le langage transformé.

\vphantom{}

Nous avons également fourni une preuve de correction pour chaque algorithme,
montrant que les automates produits reconnaissent exactement les langages
visés. Enfin, nous avons analysé la complexité en nombre d’états dans le pire
des cas.

\vphantom{}

Des perspectives intéressantes s’ouvrent à la suite de ce travail. En
particulier, affiner les bornes de complexité.


\end{document}
